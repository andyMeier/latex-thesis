\documentclass[]{article}

%packages
\usepackage{fancyhdr}
\usepackage{graphicx}
\usepackage{xcolor}
\usepackage{float}
\usepackage{booktabs}
\usepackage{subcaption}
\usepackage[format=plain, font=it]{caption}
\usepackage{array}
\newcommand{\head}[2]{\multicolumn{1}{>{\centering\arraybackslash}p{#1}}{\textbf{#2}}}
\usepackage{hyperref}
\usepackage{amsmath}

\begin{document}

\begin{titlepage}

\newcommand{\HRule}{\rule{\linewidth}{0.5mm}} % Defines a new command for the horizontal lines, change thickness here

\center % Center everything on the page
 
%----------------------------------------------------------------------------------------
%	HEADING SECTIONS
%----------------------------------------------------------------------------------------

\textsc{\LARGE University of Twente}\\[1.5cm] % Name of your university/college
\textsc{\large Master Thesis}\\[0.5cm] % Minor heading such as course title

%----------------------------------------------------------------------------------------
%	TITLE SECTION
%----------------------------------------------------------------------------------------

\HRule \\[0.4cm]
\textsc{\Large Trust in automated decision making}\\[0.5cm] % Major heading such as course name
\textsc{\large How user's trust and perceived understanding is influenced by the quality of automatically generated explanations}\\[0.5cm] % Minor heading such as course title
\HRule \\[1.5cm]
 
%----------------------------------------------------------------------------------------
%	AUTHOR SECTION
%----------------------------------------------------------------------------------------


\begin{minipage}{0.4\textwidth}
\begin{flushleft} \large
\emph{Author:}\\
Andrea \textsc{Papenmeier} % Your name
\end{flushleft}
\end{minipage}
~
\begin{minipage}{0.4\textwidth}
\begin{flushright} \large
\emph{Supervisors:} \\
Dr. Christin \textsc{Seifert} \\% Supervisor's Name
Dr. Gwenn \textsc{Englebienne} % Supervisor's Name
\end{flushright}
\end{minipage}\\[2cm]


%----------------------------------------------------------------------------------------
%	DATE SECTION
%----------------------------------------------------------------------------------------

{\large \today}\\[2cm] % Date, change the \today to a set date if you want to be precise

%----------------------------------------------------------------------------------------
%	LOGO SECTION
%----------------------------------------------------------------------------------------

\includegraphics{img/UT_logo.png}\\ % Include a department/university logo - this will require the graphicx package
 
%----------------------------------------------------------------------------------------

\vfill % Fill the rest of the page with whitespace

\end{titlepage}

\thispagestyle{empty}
\begin{abstract}
	Machine learning systems have become popular in fields such as marketing, recommender systems, financing, or data mining. While they show good performance in terms of ability to correctly classify data points, complex machine learning systems pose challenges for engineers and users. Their inherent complexity makes it impossible to easily understand their structure and behaviour in order to judge on their robustness, fairness, and the correctness of statistically learned relations between variables and classes. \textit{Explainable AI (xAI)} aims to solve these challenges by modelling explanations alongside with the classifiers. By increasing the transparency of a system, engineers and users are empowered to understand and subsequently judge the classifier's behaviour. With the General Data Protection Guideline (GDPR), companies are obligated to ensure fairness in automatic profiling or automated decision making. Discovering automated discrimination in algorithms can be done by investigating the system via explanations. Other positive effects of explainability are user trust and acceptance. Inappropriate trust, however, can have harmful consequences. In safety-critical domains such as terrorism detection or physical human-robot interaction, users should not be fooled by persuasive, yet untruthful explanations. We therefore conduct a user study in which we investigate the effects of truthfulness and algorithmic performance on user trust. Our findings show that the accuracy of a classifier is more important than its transparency for user trust. Adding an explanation for a classification result can potentially harm user trust, for example when adding nonsensical (untruthful) explanations for a classifier with good or moderate accuracy. We also find that users cannot be tricked into having trust for a bad classifier with good explanations. In this research, we also compare self-reported trust to trust measured implicitly via the user's willingness to follow a classifier's prediction. The results show conflicting evidence: While users report to have highest trust in a system with high accuracy but without explanations, they show higher willingness to accept a classifier's prediction with high accuracy and meaningful explanations.
\end{abstract}

\newpage
\thispagestyle{empty}
\tableofcontents
\setcounter{tocdepth}{1}
\newpage
\thispagestyle{empty}
\listoffigures
\newpage
\thispagestyle{empty}
\listoftables
\newpage
\setcounter{page}{1}



\pagestyle{fancy}

\pagebreak
\hspace{0pt}
\vfill
\section*{Acknowledgements}
I truly believe that creativity and ideas are never the effort of a single person. \newline
I owe profound gratitude to my supervisors, Dr. C. Seifert and Dr. G. Englebienne, who supported me throughout every stage of the project with constructive and inspiring discussions and practical advice. My grateful thanks are also extended to the Human-Media Interaction Group, and the Data Science Group of the University of Twente for supporting this research financially. I would also like to thank Dr. K.P. Truong for advising me on the measurement of trust, and Dr. R.B. Trieschnigg for encouraging me to tackle this thesis topic in the first place. Finally, I wish to thank my wonderful husband for the countless discussions and his critical mind.\newline
\vfill
\hspace{0pt}
\pagebreak
\section{Introduction}

State of the world \newline
The big BUT\newline
--- Xerox experiment [32] \newline
Therefore, we did\newline
The key findings are\newline
The contributions of this work are\newline
In HCI, the purpose of empirical contributions is to reveal formerly unknown insights about human behavior in relation to information or technology.


\section{Background}
{\color{red}---Catchy first sentence.}\newline
\textit{Machine learning} aims to infer generally valid relationships from a finite set of training data and apply those learned relations to new data \cite{domingos2012few} \cite{kotsiantis2007supervised}. While some problems can be solved by manually encoding explicit rules, others require a different approach as explicit decision-making does not deliver highly accurate results \cite{burrell2016machine}. Determining a student's grade in a multiple choice test can be solved by explicitly encoding mathematical rules, yet deciding whether the tonality of a text is positive or negative needs more than a simple rule set to function accurately \cite{melville2009sentiment}. The datasets needed to train machine learning models are often large and represented in a high-dimensional feature space, which makes it impossible for a human to carry out the learning task like a machine can. However, machines can be used to extend the cognitive capabilities of humans when working together on those learning tasks. \cite{ventocilla2018taxonomy} describes the fruitful collaboration between human and machine as \textit{augmented intelligence}, pointing at the positive aspect of machine learning support.\newline
{\color{red}---Narrowing topic to decision-making and discriminative algorithms and define "decision" as output from ML systems}


%------------------------------------------------------------------
\subsection{Interpretability in AI}
Humans cooperating with machines need to understand the principles of the method that is employed - a property referred to as \textit{transparency} \cite{kotsiantis2007supervised}. \textit{Opacity}, the direct opposite of transparency \cite{lipton2016mythos}, is a major problem for augmented intelligence. Although opacity can be used voluntarily as a means to self-protection and censorship, it also arises involuntarily due to missing technical expertise and failed human intuition and cognitive abilities \cite{burrell2016machine}.\newline
On the application-side of machine learning systems, the question of transparency brings up the notion of \textit{interpretability}. Interpretability refers to how well a ``typical classifier generated by a learning algorithm" can be understood \cite{kotsiantis2007supervised}, as compared to the theoretical principle of the method. That is, an interpretable machine learning system is either inherently interpretable, meaning that its operations and result patterns can be understood by a human \cite{biran2017explanation} \cite{ventocilla2018taxonomy}, or it is capable of generating descriptions understandable to humans \cite{gilpin2018explaining}. It is also possible to equip a system retrospectively with interpretability by adding a proxy model capable of mirroring the original system's behaviour while being comprehensible for humans \cite{guidotti2018survey}. Using an interpretable system as a human means being enabled to make inferences about underlying data \cite{ventocilla2018taxonomy}.\newline
\cite{guidotti2018survey} assigns ten desired dimensions to interpretable machine learning systems:
\begin{itemize}
	\item \textit{Scope}: Global interpretability (understanding the model and operations) and local interpretability (understanding what brought about a single decision)
	\item \textit{Timing}: Time scope available in the application use case for a target user to understand 
	\item \textit{Prior knowledge}: Level of expertise of target user
	\item \textit{Dimensionality}: Size of the model and the data
	\item \textit{Accuracy}: Target accuracy of the system while maintaining interpretability
	\item \textit{Fidelity}: Accuracy of explanation vs. accuracy of model
	\item \textit{Fairness}: Robustness against automated discrimination and ethically challenging biases in data
	\item \textit{Privacy}: Protection of sensible and personal data
	\item \textit{Monotonicity}: Level of monotonicity in relations of input and output (human intuition is largely monotonic)
	\item \textit{Usability}: Efficiency, effectiveness, and joy of use
\end{itemize}
In the context of interpretability for machine learning systems, the terms \textit{understandability}, \textit{comprehensibility}, \textit{explainability}, and \textit{justification} are often mentioned in literature. In this paper, we adopt the definition of \cite{ruping2006learning}. \textit{Understandability}, \textit{accuracy} of the explanation, and \textit{efficiency} of the explanation together form \textit{interpretability}. \textit{Explainability} is a synonym of \textit{comprehensibility} \cite{weihs2003combining}, which is also synonymic to \textit{understandability} \cite{bibal2016interpretability} and therefore an aspect of interpretability, showing the reasons for the system's behaviour \cite{gilpin2018explaining}. Figure \ref{fig:definitions} gives an overview over these terms. Finally, \textit{justification} refers to the evidence for why a decision is correct, which does not necessarily include the underlying reasons and causes \cite{biran2017explanation}.\newline
\begin{figure} [h]
	\centering
	\includegraphics[width=0.7\linewidth]{img/definitions}
	\caption{Relation of terms connected to interpretability}
	\label{fig:definitions}
\end{figure}
If the human cognition is augmented by a machine learning system, talking about interpretability should also include discussing the interpretability of the human in the loop. \cite{lipton2016mythos} argues that human behaviour is often mistakenly identified as interpretable because humans can explain their actions and beliefs. Yet the actual operations of the human brain remain opaque, which contradicts the concept of interpretability \cite{lipton2016mythos}. If human interpretability is taken as a point of reference for the discussion of algorithmic interpretability, \cite{lipton2016mythos}'s argument should be taken into account. Human interpretability, however, is not the focus of this paper and will therefore not be discussed in more detail here.\newline





%------------------------------------------------------------------
\subsection{Need for Explainability in AI}
\label{subsec:explainability}
A subfield of artificial intelligence research revolves solely around the explainability of intelligent systems: \textit{xAI}, explainable artificial intelligence, for the purpose of enabling communication with agents about their reasoning \cite{hendricks2018generating}. xAI systems face a trade-off challenge: Their explanation has to be complete and interpretable at the same time \cite{gilpin2018explaining}. The attention span and cognitive abilities of humans therefore become an important factor to consider in the design of a xAI systemm \cite{kulesza2013too}. Furthermore, the goal of explaining the system is twofold: create actual knowledge and convince the user that the knowledge is sound and complete. Actual understanding and perceived understanding however do not always go hand in hand: Persuasive systems can convince the user without creating actual transparency \cite{gilpin2018explaining}. The persuasiveness of an explanation is uncoupled from the actual information content of an explanation \cite{biran2017explanation} and needs to be taken into account in user studies. As users can only report on their perception of the explanation, an objective measure to evaluate the fidelity of an explanation is needed. High-fidelity (also called descriptive) explanations are faithful, in that they represent truthful information about the underlying machine learning model \cite{herman2017promise}. Persuasive explanations, on the opposite, are less faithful to the underlying model, yet open up possibilities for abstraction, simplification, analogies, and other stylistic devices for communication. \cite{herman2017promise} notes a dilemma in explanation fidelity: ``This freedom permits explanations better tailored to human cognitive function, making them more functionally interpretable", but ``descriptive explanations best satisfy the ethical goal of transparency". The xAI practitioner therefore needs to consider a tradeoff between fidelity and interpretability. \newline 
Besides low-fidelity persuasiveness, badly designed explanations likewise ``provide an understanding that is at best incomplete and at worst false reassurance" \cite{burrell2016machine}. Therefore, not only possible explanations for white box (inherently interpretable) and black box (inherently non-interpretable) systems need to be examined, but also the (visual) design and communication of explanations \cite{guidotti2018survey}. \newline
In recent years, machine learning algorithms employed in the wild show a trend towards increasing accuracy but also increasing complexity. In general, the higher the accuracy and complexity, the lower the explainability \cite{richardson2018survey} \cite{chen2018learning} in machine learning. However, users do not necessarily perceive systems with simple explanations as more understandable \cite{allahyari2011user}. The authors of the user study in \cite{allahyari2011user} hypothesise that users detect missing information in simple explanations, which in turn leads to the perception of incomprehensibility. \cite{van2018contrastive} examined user preferences in more detail and concluded that users overall preferred more soundness and completeness over simplicity, as well as global explanations over local explanations.\newline
Humans involved in the explanation process are not only users, but also domain experts and engineers during the design and training phase. As explanations are user-dependent (not monolithic) \cite{preece2018asking}, the design and evaluation of explanation needs to be conducted in reference to the target users. Including experts in the modelling and training process is not only a way to integrate expert knowledge that is otherwise difficult to model, but can also increase user trust \cite{ventocilla2018taxonomy}. \cite{liu2017towards} call the situation where a human expert works alongside the machine learning system to improve it ``mixed initiative guidance". \newline

\subsubsection{Explanation Goals}
\label{subsubsec:Explanation Goals}
Machine learning systems are able to achieve high accuracy on classification tasks, for example in information retrieval, data mining, speech recognition, and computer graphics \cite{liu2017towards}. Explainability is a means to ensure that machine learning systems are not only right in a high number of cases, but right for the right reasons \cite{preece2018asking}. High accuracy does not necessarily mean that correct generalisations were learned from the dataset or that no biases were present in the data.\newline
The need for interpretability is dependent on the role of the explanation user and the severity of the consequences of the classification result and possible errors. Since explanations are not monolithic, i.e. have to be adapted to the target user's level of expertise, preferences for explanation types, and cognitive capabilities, the need for interpretability is also dependent on the targeted audience. Furthermore, different users can have different data access rights and have different goals to achieve in their interaction with the system \cite{vorm2018assessing}. While an engineer could be interested in technical details, a bank employee assessing loan credibility could be interested in similar cases and relevant characteristics of a single decision case. \cite{richardson2018survey} separates a general need for interpretability into three categories:
\begin{itemize}
	\item \textbf{no need} for interpretability if no consequences arise from faulty decisions
	\item interpretability is \textbf{beneficial} if consequences for individuals arise from faulty decisions
	\item interpretability is \textbf{critical} if serious consequences arise from faulty decisions
\end{itemize}
The three classes of interpretability needs give an overview about possible consequences, yet are too general to serve as guideline for practitioners. More details about decisive factors are needed.\newline
For users of an automatic decision system, having insights into the system functioning and decision process increases trust \cite{preece2018asking} \cite{diakopoulos2016accountability} \cite{biran2017explanation} \cite{cramer2008effects} \cite{vorm2018assessing}, even in critical decisions such as medical diagnosis \cite{allahyari2011user}. The level of trust should be in relation to the soundness and completeness of an explanation. Having too much or too little trust in a system can hinder fruitful interaction between the user and the system \cite{preece2018asking} \cite{richardson2018survey} \cite{van2018contrastive} \cite{ribeiro2016should}. Other positive effects on users are satisfaction and acceptance \cite{biran2017explanation} \cite{cramer2008effects} \cite{vorm2018assessing} as well as the ability to predict the system's performance correctly \cite{biran2017explanation}.\newline
\cite{liu2017towards} identifies three goals of explanability in machine learning:
\begin{itemize}
	\item \textit{Understanding and reassurance}: right for the right reasons
	\item \textit{Diagnosis}: analysis of errors, unacceptable performance, or behaviour
	\item \textit{Refinement}: improving robustness and performance
\end{itemize}
From the point of view of engineers and experts, explanations help to design, debug, and improve an automatic decision system \cite{preece2018asking}. Explanations facilitate the identification of unintuitive, systematic errors \cite{gilpin2018explaining} \cite{ribeiro2016should} in the design and redundantise time-consuming trial-and-error procedures for parameter optimisation \cite{liu2017towards}. Unethical biases in training data leading to automated discrimination \cite{diakopoulos2016accountability} can be identified and examined via explanations \cite{gilpin2018explaining} \cite{richardson2018survey} \cite{ribeiro2016should}. Ultimately, the early identification of errors avoids costly errors in high-risk domains \cite{diakopoulos2016accountability} \cite{bibal2016interpretability} \cite{van2018contrastive} and ensures human safety in safety-critical tasks \cite{gilpin2018explaining} \cite{richardson2018survey}.\newline
Besides helping users and engineers, explanations also serve general goals of protection, conformity, and knowledge management. Criminals or hackers that aim to disturb the system or take advantage of it can make imperceptible changes to the input data or model at hidden levels. Having a system capable of explaining its behaviour and inner structure helps to identify unwanted alterations \cite{gilpin2018explaining}. With the European General Data Prodection Regulation (GDPR) put into place in 2018, a debate on a \textit{right to explanation} started, which will be discussed in the following section. Although the specific implications of the right to explanation remain unclear, it should still be noted that designing interpretability follows up on that regulation \cite{goodman16eu} \cite{gilpin2018explaining} \cite{bibal2016interpretability}. Finally, the most general goal of implementing explanations for automatic decision systems is the opening and accessibility of a knowledge source \cite{bibal2016interpretability} \cite{richardson2018survey}. The relations derived by a machine learner (stored in the model) can deliver relevant knowledge about the data at hand.\newline


\subsubsection{Regulations and Accountability}
The General Data Protection Regulation (GDPR) is a European law dealing with the processing of personal data within the European Economic Area (EEA, includes also all countries of the EU). The law holds for all companies within the EEA, companies with subsidiaries in the EEA, and any company processing personal data of a citizen of the EEA. In this context, ``processing" does not only relate to automatic systems but also spans to manual processing of personal data \cite{goodman16eu}. The GDPR defines personal data as data relating to an identifiable natural person, i.e. data that can be used to identify a person {\color{red}[REF TO LAW TEXT]}. Names, location data, or personal identification numbers are all examples of personal data that falls under the GDPR. \cite{goodman16eu} identifies two consequences of the GDPR: the legal right to non-discrimination, and a right to explanation.\newline
Algorithmic decisions must not be based on sensitive, personal data (GDPR article 22 paragraph 4) that are nowadays used to identify groups of people with similar characteristics, such as ethnicity, religion, gender, disability, sexuality, and more \cite{diakopoulos2016accountability}. Sensitive information can, however, correlate with non-sensitive data. Real-life data almost always reflects a society's structures and biases - explicitly through sensitive information, or implicitly via dependent information. As the task of classification means separating single instances into groups based on the available data, the biases are recovered in the model \cite{goodman16eu}. A guarantee non-discrimination is therefore difficult to achieve. The GDPR does not specify whether only sensitive data or also correlated variables have to be considered when following the law. \cite{goodman16eu} identifies both interpretations as possible.\newline
While article 13 of the GDPR specifies a right to obtain information about one's personal information and the processing of that personal information, it assures ``meaningful information about the logic involved" in profiling without further defining meaningfulness. Based on the ambiguity of ``meaningful", several interpretations exist, ranging from denial of the ``right to explanation" \cite{wachter2017right} to a positive interpretation \cite{selbst2017meaningful}. In summary, precedents are needed to clarify the boundaries of the law.\newline
Besides legal regulations, ethical considerations also play a role in augmented intelligence. Accountability is the ethical value of acknowledging responsibility for decisions and actions towards another party \cite{baldoni2016computational}. It is an inherent factor in human-human interaction; artificial intelligence employed to interact with humans or collaborate with humans in augmented intelligence settings therefore bring about the challenge of ``computational accountability" \cite{baldoni2016computational}. It is important to note that accountability is not a general issue in the digital world: For something to be held accountable of its own decisions or actions, it needs to act autonomously {\color{red}[REF WENT MISSING; CHECK AGAIN IN NOTES]}. In order to determine autonomy of an algorithm and work towards accountability, \cite{diakopoulos2016accountability} suggests to disclose the following information for machine learning systems: 
\begin{itemize}
	\item \textit{Human involvement}: who controls the algorithm, who designed it etc., leading to control through social pressure
	\item \textit{Data statistics}: accuracy, completeness, uncertainty, representativeness, labelling \& collection process, preprocessing of data
	\item \textit{Model}: input, weights, parameters, hidden information
	\item \textit{Inferencing}: covariance matrix to estimate risk, prevention measures for known errors, confidence score 
	\item \textit{Algorithmic presence}: visibility, filtering, reach of algorithm
\end{itemize}
\cite{baldoni2016computational} argues that causality is a necessary prerequisite for accountability. Machine learning algorithms often learn statistical relations between input features, which at best leads to probabilistic causality, but not certainly to deterministic causality. Whether an automatic decision system itself can be held accountable for its decisions is therefore debatable.





%------------------------------------------------------------------
\subsubsection{Application Areas}
Artificial intelligence and machine learning algorithms are nowadays employed in a variety of areas. As described in \ref{subsubsec:Explanation Goals}, the need for interpretability depends on the potential consequences of the decisions made by an automatic system. \cite{burrell2016machine} summarises the application area as all systems with ``socially consequential mechanisms of classification and ranking", pointing in particular to the consequences for humans. A similar view is expressed in \cite{poursabzi2017manipulating} and \cite{ribeiro2016should}, while \cite{guidotti2018survey} restricts the application areas in need for interpretability to those that process sensitive, i.e. personal data. In more detail, the following areas in need of interpretable intelligent systems are mentioned in literature:
\begin{itemize}
	\item \textit{Societal safety}: criminal justice [52] [19], terrorism detection [24]	
	\item \textit{Processing sensitive data}: banking, e.g. loans [52] [6] [19] [33] [36], medicine \& health data [52] [3] [1] [15] [19] [16] [24], insurances [3] [33] [36], navigation [1]
	\item \textit{Physical safety}: autonomous robotics [3] [15]
	\item \textit{Knowledge}: education [16], knowledge discovery in research [3]	
	\item \textit{Economy}: manufacturing [16], individual performance monitoring [1], economic situation analysis [1], marketing [6] [33] [36]	
\end{itemize}
But not only systems treating personal data or interacting directly with humans profit from interpretability- \cite{ventocilla2018taxonomy} suggest all machine learning based support systems as suitable candidates for interpretability. Machine learning is already employed in IT-services such as spam detection and search engines \cite{burrell2016machine} \cite{domingos2012few}, as well as in recommender systems \cite{gilpin2018explaining} \cite{richardson2018survey}.\newline
In the past, several machine learning systems have failed due to undetected systematic errors or automated discrimination. \cite{guidotti2018survey} lists incidents with machine learning systems, ranging from discrimination in the job application procedure and faulty target identification in automated weapons due to training data biases, to high differences in mortgage decisions by banks.\newline
An interesting case is the American COMPAS system for automated crime prediction. The system predicted a significantly higher relapse rate for black convicts than for whites, which is assumed to result from human bias in the training data \cite{guidotti2018survey}. The argument of human bias is often used to object the perceived impartiality of computer systems, and other examples of discrimination of ethnic minorities exist \cite{guidotti2018survey}, yet \cite{skeem2016risk} counter-argues that differences found in the data set possibly reflect actual differences existing in the real world - which would shift the discussion about auto-discrimination to the field of ethics. Furthermore, the goal of profiling and classification is to separate a data set into groups \cite{goodman16eu}; discrimination is therefore ``at some level inherent to profiling" \cite{datta2015automated}.\newline
In a study of 600.000 advertisements delivered by Google, \cite{datta2015automated} found a bias against women. Advertisements of higher-paid jobs were more often shown to men than they were to women. Google's targeted advertisements make use of profiling, i.e. delivering content to users depending on their gender, age, income, location, and other characteristics. In the study, the researchers did not have access to the algorithm and can therefore not determine whether the bias was introduced with the data set, the model, or simply by conforming to the advertisement client's requirement for profiling.\newline
Besides biased training data, systematic modelling errors can account for failures of machine learning systems. Google Flue Trends predicted the amount of humans infected with flue based on the received search queries, leading to large overestimates of actual flue cases \cite{preece2018asking}. \cite{shepperd2014researcher} investigated the work of different research groups on the same data set, finding that the main reason for variance in results originates from the composition of the group. Compared to the group composition, the choice of classifier accounted for minor variance. They therefore concluded that the human bias in machine learning systems is the main factor influencing the results.\newline
Deciding whether an automatic decision system meets legal and ethical standards requires knowledge about the system. In the case of Google's targeted advertisements, it is impossible to determine if the algorithm is discriminating women on purpose due to advertiser's requirements, or if the system has internal flaws that lead to unfair treatment. With the GDPR, judging the fairness of an automatic system is not only a concern of the company using machine learning techniques, but also the right of any data subject in the training set and the application. 



%------------------------------------------------------------------
\subsection{Explanations}
In the previous sections, we used ``explanations" as a generic term. In this section, the concept of an explanation is described in more detail.\newline
In general, an explanation is one or more reasons or justification for an action or belief \cite{preece2018asking}. Humans need explanations to build up knowledge about events, evaluate events, and ultimately to take control of the course of events.\newline
When being confronted with a new event, artifact, or information in general, humans start building internal models. These mental models are not necessarily truthful nor complete, but represent an individual's interpretation about the event. Explanations are a tool to build and refine the inner knowledge model \cite{miller2017explanation}.\newline
Explanations also help to assess events that are happening: We are able to compare methods or events with each other, justify the outcome of an event, and assign responsibility and guilt for past events \cite{miller2017explanation} \cite{keil2006explanation}. Explanations also serve to persuade someone of a belief \cite{miller2017explanation}, and can lead to appreciation through understanding \cite{keil2006explanation}.\newline
Having understood what brings a certain event about, humans can use their knowledge model to predict the consequences of (similar) events in the future \cite{miller2017explanation}. For an engineer working on a machine learning system, understanding underlying principles and consequences of the system's behaviour is a necessary step in designing a system that is ``right for the right reasons" \cite{preece2018asking}. Similarly, the knowledge model can serve to prevent unwanted states or events, restore wanted states, and reproduce observed states or events \cite{keil2006explanation}.


\subsubsection{Human-Human Explanations}
Humans build mental models of the world, an inner, mental representation of events or elements. It might be noteworthy to point out the difference between the inner knowledge model and an explanation. The mental model is a subjective set of relations resulting from an individual's thought process. An explanation, however, is the interpretation of such relations \cite{keil2006explanation}. Both the mental model and an explanation do not have to be truthful to the real world. We do not need to have complete, holistic mental models in order to use an artifact, but a \textit{functional} model is needed to tell us how to use and make use of it, while a \textit{structural} model stores information about the composition and how it is built \cite{kulesza2013too}.\newline
Explanations are a cognitive and social process: The challenge of explaining includes finding a complete but compressed explanation, and transferring the explanation from the explainer to the explainee \cite{miller2017explanation}. In its purest sense, ``complete" means an explanation that uncovers all relevant causes \cite{miller2017explanation}, which is rarely the case in the real world.\newline
\cite{keil2006explanation} summarises four aspects of explanations:
\begin{itemize}
	\item \textit{Causal pattern content}: an explanation can reveal information about a common cause with several effects, a common effect brought about by several causes, a linear chain of events influencing each other chronologically, or causes that relate to the inner state of living things (homeostatics), e.g. intent
	\item \textit{Explanatory stance}: refers to the mechanics, the design, and intention \cite{miller2017explanation}. Atypical explanatory stances can lead to distorted understanding.
	\item \textit{Explanatory domain}: different fields have different preferences of explanation stances
	\item \textit{Social-emotional content}: can alter acceptance threshold and influence recipient's perception of explained event 
\end{itemize}
What constitutes a good explanation? \cite{keil2006explanation} describes good explanations as being non-circular, showing coherence, and having a high relevance for the recipient. Circularity are causal chains where an effect is given as cause to itself (with zero or more causal steps in between). Explanations can, but do not have to, explain causal relations \cite{keil2006explanation}. Especially in the case of machine learning algorithms, the learned model shows correlation, not causation. Explanations for statistical models therefore cannot draw on typical causal explanations as found in human-human communication {\color{red}[REF NEEDED]}. The probabilistic interpretation of causality comes closest to the patterns learned in statistical models: If an even $A$ caused an event $B$, then the occurrence of $A$ increases the probability of $B$ occurring. Statistical facts are not satisfactory elements of an explanation, unless explaining the event of observing a fact \cite{miller2017explanation}. Arguably, this holds true for statistical learning. Coherence refers to the systemacity of explanation elements: good explanations do not hold contradicting elements, but elements that influence each other \cite{keil2006explanation}. Finally, relevance is driven by the level of detail given in the explanation. The sender has to adapt the explanation to the recipient's prior knowledge level and cognitive ability to understand the explanation \cite{miller2017explanation}, which can mean to generalise and to omit information - \cite{keil2006explanation} calls this adaptation process the ``common informational grounding". The act of explaining also includes a broader grounding of shared beliefs and meanings of events and the world \cite{miller2017explanation}. The ``compression problem" poses a major challenge in constructing explanations for humans. Humans tend to not comprise all possible causes and aspects of the high-dimensional real world in an explanation, suggesting that there are compression strategies (on the sender's side) and coping strategies (on the recipient's side) in place \cite{keil2006explanation}. \newline
\cite{miller2017explanation} notes that besides presenting likely causes, and coherence, a good explanation is simple and general. The latter two characteristics refer to the agreement widely accepted in science that a simple theory (or, in this case, an explanation) is favoured over a more complicated theory if both explain an equal set of events or states.\newline
\cite{kulesza2013too} defines a good explanation as sound, complete, but not overwhelming. While soundness refers to the level of truthfulness, completeness describes the level of disclosure \cite{kulesza2013too}. In order to avoid overwhelming the explainee, the informational grounding process takes place, i.e. a common understanding of related elements and an adaptation of the explanation's detailedness to the explainee's knowledge level. In general, the more diverse the given evidence, the higher the recipient's acceptance of the explanation \cite{keil2006explanation}.\newline
Explainees' cultural background is known to influence their preference for an explanation type - explaining foremost the mechanics, the design, or the intention of an event or artifact. Although different explanation types are preferred in different cultures, all explanation types can be understood by all cultures in general \cite{keil2006explanation}.\newline
An experiment by \cite{langer1978mindlessness} shows that humans have behavioural \textit{scripts} in place when confronted with an explanation. The pure presence of an explanation, regardless of the informational content, can make a difference in how people react to requests. In the experiment, people busy with making copies at a copy machine were asked to let another person go ahead. Three conditions were examined: issuing the request of skipping line with a reasonable explanation (``because I am in a rush"), with placebic information (using the structure of an explanation without giving actual explanatory information: ``because I need to make copies"), and without any explanation. The compliance rate for cases without any explanation was significantly lower than the compliance in cases where any kind of explanation (placebic or informative) was given, with little difference between the two explanation types \cite{langer1978mindlessness}. \cite{weller2017challenges} points out the advantage of such explanation - no matter the informative content -: ``[t]o make a user (the audience) feel comfortable with a prediction or decision so that they keep using the system". \cite{langer1978mindlessness} explains this behaviour with behavioural scripts that are triggered when people find themselves in a state of \textit{mindlessness}. In a mindless state, the automatic script ``comply if reason is given" is triggered, no matter what the reason is. The mindless state, however, is revoked if the consequences of complying become more severe. In an attentive state, the explanation does make a difference: People were more likely to comply when an informative explanation was given, as compared to the placebic explanation \cite{langer1978mindlessness}. 



\subsubsection{AI-Human Explanations}
Understanding what brought about a machine learning decision can be complex. For explaining the reasons that led to a specific classification, or the classifier in general, different aspects can be highlighted.\newline
{\color{blue}Do I need to introduce machine learning systems here? E.g. explain basic concepts like the dataset, features, the algorithm, model, etc? Or can we assume that our readers are already familiar with such concepts?}\newline
A machine learning system generating automatic decisions contains five elements \cite{ventocilla2018taxonomy}:
\begin{itemize}
	\item Dataset and subsequent features
	\item Optimizer or learning algorithm
	\item Model 
	\item Prediction, or more generally, the result
	\item Evaluator
\end{itemize}
All elements are chosen or designed by the system engineer and hence 
\cite{biran2017explanation} categorises aspects of a machine learning decisions and respective explanation suggestions into three layers:
\begin{itemize}
	\item \textit{Feature-level}: feature meaning and influence, actual vs. expected contribution per feature
	\item \textit{Sample-level}: explanation vector, linguistic explanation for textual data using bag-of-words, subtext as justification for class (trained independently), caption generation (similar to image captions) 
	\item \textit{Model-level}: rule extraction, prototypes \& criticism samples representing model, proxy model (inherently interpretable) with comparable accuracy (author's note: supposedly meant comparable decision generation, not simple accuracy)
\end{itemize}
The categories from \cite{biran2017explanation} make a distinction between the input (feature-level), a local explanation focussing on a single instance (sample-level), and a global view that comprises the whole model and its behaviour (model-level). While those aspects focus rather on the artifacts that play a role in automated decision systems, others divide the aspects of explainable AI systems based on the processes and steps \cite{bibal2016interpretability} \cite{gilpin2018explaining} \cite{miller2017explanation} \cite{preece2018asking} \cite{richardson2018survey} \cite{ventocilla2018taxonomy}:
\begin{itemize}
	\item \textit{Data \& features}: representation of data 
	\item \textit{Operations}: processing of data, computations, learning algorithm
	\item \textit{Model}: parameters, representation
	\item \textit{Prediction}: visualisation, e.g. heat maps
	\item \textit{Secondary / add-on system}: generation of explanation via behaviour, learning algorithm behaviour
\end{itemize}
\cite{richardson2018survey} stress that different explainability needs call for different timings of the explanation. Showing the explanation \textbf{before} a classification or generation task is useful for justifying the next step or explaining the plan. \textbf{During} a task, information about the operations and features can help identifying errors for correction and foster trust. Explaining the results of a task \textbf{after} the process is useful for reporting and knowledge discovery.



single focus: feature-based explanation best for recommender systems (as compared to similar previous decisions and similar neighbor decisions) [10] \newline


\textbf{Explanation selection}: it is not possible to show every case, parameter, feature importance to the user, therefore a selection of exemplary cases needs to be made [24]. Global explanation can originate from a set of representative cases [24].



\subsubsection{When to explain?}
[15] stresses that different explainability needs call for different timings of the explanation. Showing the explanation \textbf{before} a classification or generation task is useful for justifying the next step or explaining the plan. {\color{green}\textbf{During} a task, information about the operations and features can help identifying errors for correction and foster trust.} Explaining the results of a task \textbf{after} the process is useful for reporting and knowledge discovery.



\subsubsection{Explanation Systems}
For models that are not inherently interpretable, the explanation can only be an approximation and cannot be complete (definition of non interpretable) \cite{miller2017explanation}. There can be approximations for the computation / operations detecting properties and categorisations, and approximations of the decision behaviour \cite{miller2017explanation}.\newline

counterfactual explanation [12] with fact \& foil \newline
[4] for overview over solutions for understanding, diagnosis, refinement \newline
[6] for overview of solutions for explaining features, operations, generative explanations \newline
[16] for solutions for dataset, optimizer, model, predictor and evaluator \newline
[14] for set of programs (MYCIN, NEOMYCIN, CENTAUR, EES) that try to model explanations alongside with system \newline
[19] presenting the L2X system\newline
[24] Explanation software: LIME, ELUCIDEBUG

For feature-based models, [19] suggests salience map masks on input features, comparable cases (input and output) as reference (or very dissimilar cases as counterfactuals), and mutual information analysis per feature. For the latter, they use the Kullback-Leibler divergence to calculate the mutual information of two vectors: Learning to explain (L2X).\newline

Inherently interpretable / transparent models:
\begin{itemize}
	\item decision trees (graphical representation), rules (textual representation), linear models (feature magnitude and sign) [3]
	\item shallow rule-based models, decision lists, decision trees, feature selection, compositional generative models [10]
	\item decision trees, Naive Bayes, Rule-Learners [71]
\end{itemize}

[{\color{red}REF NEEDED}] add-on and post-hoc systems might be good as explaining, but this fact in itself does not guarantee a sound, i.e. truthful, explanation, ``however plausible they appear" [31]. \newline

[15] suggests to develop a new class of learning algorithms that have an inherent ``explainability hyperparameter" to achieve high accuracy AND high explainability.\newline

[36] argues that most high-dimensional real-world application data is ``concentrated on or near a lower-dimensional manifold" [36], dimension reduction techniques like PCA or other feature selection algorithms can therefore be used to overcome the curse of dimensionality. \newline

\textbf{explanations for texts}:
[7] solution to recent development in text mining, where texts are represented in a high-dimensional vector space (e.g. fasttext, word2vec) and classified with neural nets. Compared to BOW/SVM, the W2V/CCN they used yields equally good results, because the CNN is better at identifying characteristic words.\newline
[19] designed a system that uses deep neural networks for classification and mutual information for getting the input feature importance (in their case, single words). \newline

\textbf{Relevant words}: A word is relevant to the text if removing it from the texts and classifying again results in a decrease of the classification score across all texts
[56] take the opposite approach by eliminating irrelevant words, which leaves the relevant ones but show that this method does not work for neural classifiers






%------------------------------------------------------------------
\subsubsection{Explanation Evaluation}
[6]:
\begin{itemize}
	\item application grounded: true context, true task, users
	\item human-grounded: usability tests, human performance tests
	\item functionally grounded: no users, proxy
\end{itemize}
[8] evaluation of model interpretability:
\begin{itemize}
	\item heuristics: number of rules, number of nodes, minimum description length (model parameters)
	\item generics: ability to select features, ability to produce class-typical data points, ability to provide information about decision boundaries
	\item specifics: user testing / perception (BUT: evaluation of visuals and perceived model rather than actual model), e.g. by measuring accuracy of prediction, answer time, answer confidence, understanding of model
\end{itemize}
[15] rather combination than only a single one:
\begin{itemize}
	\item algorithm performance score
	\item user performance score
	\item user satisfaction score 
\end{itemize}







%------------------------------------------------------------------
\subsection{Trust in AI}
[25] notes that there exists no precise definition of trust in the field of computer science\newline

[TRUST 02] examined the concept of trust in close relationships and define it as the willingness to put oneself at a risk and believing that the other will be benevolent. They grouped aspects of interpersonal trust into a model with three components: faith, dependability, predictability [TRUST 02]. \newline
Placed in agent, not a characteristic inherent to an agent [TRUST 02]\newline
Trust is a subjective experience rather than objectively measurable [TRUST 05] [23]. \newline
dynamic: evolves as relationship matures [TRUST 02]\newline
attribution of characteristics, e.g. dependability (repeated confirmation in risky situations), reliability (consistency or recurrent behaviour) [TRUST 02]\newline
inappropriate trust can be harmful [17]\newline
Trust as experience, trustworthiness is the characteristic and in case of computer programs consists of factors such as security, privacy, dependability, usability, correctness [TRUST 05] [TRUST 06]. Trust relates to the assurance that a system performs as expected [TRUST 05]. \newline
Trust in a system can be misused: e-crime with negative side effects, e.g. data misuse [TRUST 05]. \newline 

\subsubsection{Gaining User Trust}
Trust factors: appeal, competence (privacy, security, functionality), transparency, duration (relationship, affiliation), reputation [23] \newline
Concerning algorithms, users can put global trust into the system, which means trusting the model itself. Trust can also be assigned locally, into an individual decision. [24] \newline
Trust dimensions of web systems: target (the entity being evaluated), representation (encoding of trust via social warranty, certificates, etc.), method (security), management (the entity putting trust into the system), computation (evaluation metric), purpose [25] \newline
For classification: expectation mismatch leads to direct decrease in trust [30], strength of decrease depends on the type of mismatch. Data-related mismatch weights less strongly than logic-driven mismatch. [30] \newline 
[31] argues that trust in machine learning algorithms also depends on the characteristics of misclassified cases. He points out that an automatic system can be considered trustworthy if it behaves exactly like humans, i.e. it misclassifies the same data points as a human and is correct on those cases that a human would also correctly classify [31].

\subsubsection{Trust Evaluation}
[23]: using experts to assign a weighted label to each element on a website or GUI and calculating a score
\begin{itemize}
	\item [-1] irritant
	\item [1] chaotic
	\item [2] assuring
	\item [3] motivating
	\item [0] not present
\end{itemize}
But user study showed that experts find it problematic to assign discrete trust values. The advantage of this approach, however, is that it is possible to compare multiple websites [23]. \newline
user study with closed and open questions [24]:
\begin{itemize}
	\item Do you trust this algorithm to work well in the real world?
	\item Why do you trust this algorithm to work well in the real world?
	\item How do you think the algorithm distinguished between the two classes?
	\item How certain are you of the correctness of your explanation? 
\end{itemize}
[TRUST 02] develops a trust scale with 26 items, each belonging to one of the three trust factors (faith, dependability, predictability). \newline
[TRUST 01] describes online trust (websites) as developing from external factors (website's reputation, navigational architecture, user's prior experience) as well as perceived factors (credibility, ease of use, risk) \newline
{\color{green}``willingness to accept a computer-generated recommendation is considered a proxy measure of trust" [38] }

\subsubsection{Perceived Understanding}
Perceived understanding important for trust (rather than actual understanding):\newline
``Findings show that the transparent version was perceived as more understandable and perceived understanding correlated with perceived competence, trust and acceptance of the system. Future research is necessary to evaluate the effects of transparency on trust in and acceptance of user-adaptive systems" [59] \newline
Most questionnaires use factual statements to investigate perceived understanding. Participants rate the statements according to their confidence of understanding [UND 03] [UND 07] or directly their subjective understanding [UND 01] [UND 02] [UND 04] [UND 05]






%------------------------------------------------------------------
\subsection{Summary}
Summary\newline
\begin{itemize}
	\item summary 
	\item systems
	\item evaluation of explanations and of trust
\end{itemize}
Hypotheses














\section{Method}
Intro

\subsection{Use Case Scenario}
definition of offensive language [34] \newline
hate speech detection systems \newline








\section{Implementation}

Intro


\subsection{Dataset Selection}
Few datasets with offensive language texts are publicly available. Table \ref{tab:StatsAllDatasets} presents an overview of four available datasets, their sizes and class balances. 
%-------------------------------------------------------
\begin{table}[!ht]
	\centering
	\begin{tabular}{llll}
		\textbf{Corpus} & \textbf{Size} & \textbf{Classes} & \textbf{} \\ \hline
		Davidson\footnote{https://github.com/t-davidson/hate-speech-and-offensive-language} & 25,000 & \begin{tabular}[c]{@{}l@{}} hate speech\\offensive\\neither\end{tabular} & \begin{tabular}[c]{@{}l@{}} 6\%\\77\%\\17\%\end{tabular} \\ \hline
		Imperium\footnote{https://www.kaggle.com/c/detecting-insults-in-social-commentary/data} & 3,947 & \begin{tabular}[c]{@{}l@{}} neutral\\insulting\end{tabular} & \begin{tabular}[c]{@{}l@{}} 73\%\\27\%\end{tabular} \\ \hline
		Analytics Vidhya\footnote{https://datahack.analyticsvidhya.com/contest/practice-problem-twitter-sentiment-analysis/} & 31,962 & \begin{tabular}[c]{@{}l@{}} hate speech\\no hate speech\end{tabular} & \begin{tabular}[c]{@{}l@{}} 7\%\\93\%\end{tabular} \\ \hline
		SwissText\footnote{https://www.swisstext.org/workshops/2018/Hackathon.html} & 159,570 & \begin{tabular}[c]{@{}l@{}} toxic\\severe\_toxic\\obscene\\threat\\insult\\hate speech\\neither\end{tabular} & \begin{tabular}[c]{@{}l@{}} 10\%\\1\%\\5\%\\0.3\%\\5\%\\1\%\\72.7\%\end{tabular} \\ \hline        
	\end{tabular}
	\caption{Publicly available datasets for offensive language texts}
	\label{tab:StatsAllDatasets}
\end{table}
%-------------------------------------------------------
While the dataset of SwissText has the most fine-grained labelling of its data points, details on how the labels were assigned (i.e. number of annotators, inter-annotator agreement score, definition of the classes) are not available. The same holds for the datasets of Analytics Vidhya and Imperium.\newline
In contrast, Davidson's datasets comes with a description of how the data points were collected, how the classes are defined, and uses at least three annotators per text. Furthermore, Davidson's dataset contains the most data points labelled as offensive: roughly 20750 Tweets fall into this category, while the Analytics Vidhya dataset contains 2240 hate speech texts, SwissText 1600, and Imperium 1000.\newline
Throughout the literature, different definitions of hate speech and offensive language are given. For using a dataset in a user study with the scenario of a social media administrator, the definition of the label has to be clear. We therefore chose to work with the dataset of Davidson et al., as it offers the most detailed description of its labels and how the labels were obtained.


\subsection{Dataset Construction}
The original dataset was collected by Davidson et al. \cite{davidson2017automated} for their research on defining and differentiating hate speech from offensive language. They constructed a dataset with offensive Tweets and hate speech by conducting a keyword search on Twitter, using keywords registered in the hatebase dictionary\footnote{https://www.hatebase.org}. The timelines of Twitter users identified with the keyword search were scraped, resulting in a dataset of over 8 million Tweets. They selected 25 000 Tweets at random and had at least 3 annotators from Figure Eight\footnote{https://www.figure-eight.com} (formerly Crowd Flower) who labelled each Tweet as containing hate speech, offensive language, or neither. They reached an inter-annotator agreement of 0.92 \cite{davidson2017automated}. The dataset is publicly available on GitHub\footnote{https://github.com/t-davidson/hate-speech-and-offensive-language}.\newline
The biggest class in the dataset are the offensive language Tweets (77\%), while non-offensive Tweets represent 17\%, and hate speech 6\% of the dataset. \newline
For our research, we are only interested in offensive and not offensive Tweets. We therefore excluded Tweets labelled as hate speech for the further construction of our dataset. We produced a balanced dataset by selecting only Tweets with the maximum inter-annotator agreement from each of the two remaining classes, and randomly drew Tweets from the bigger class (offensive Tweets) until the size of the subset was equal to the size of the smaller class (non-offensive Tweets). Table \ref{tab:StatsDataset} presents statistical information about the resulting dataset.
%-------------------------------------------------------
\begin{table}[!ht]
	\centering
	\begin{tabular}{lll}
		\hline
		\textbf{} & \textbf{Not Offensive Class} & \textbf{Offensive Class} \\ \hline
		Size (absolute) & 4,162 & 4,162 \\
		Size (relative) & 50.00\% & 50.00\% \\
		Total words & 58,288 & 61,504 \\
		Unique words & 6,437 & 9,855 \\
		\begin{tabular}[c]{@{}l@{}}Average words\\per Tweet\end{tabular} & 14.00 & 14.78 \\ \hline         
	\end{tabular}
	\caption{Statistical characteristics of the constructed dataset}
	\label{tab:StatsDataset}
\end{table}
%-------------------------------------------------------




\subsection{Dataset Preprocessing}
Tweets exhibit some special characteristics. First, the maximum length of a single Tweet is 140 characters. Twitter doubled the length in November 2017, yet the dataset was collected before this data and therefore contains only Tweets of 140 characters or shorter. Twitter users found creative ways to make use of the 140 characters given, leading to the usage of short URLs instead of original URLs \cite{xiang2012detecting}, intentional reductions of words (e.g. ``nite" instead of ``night") \cite{xiang2012detecting}, abbreviations \cite{gupta2018proposed}, emojis \cite{ghorai2016information} \cite{watanabe2018hate} and smilies \cite{smailovic2013predictive} \cite{hovelmann2017fasttext}.\newline
Furthermore, social media content can be unstructured, with word creations that are non in standard dictionaries, like slang words \cite{gupta2018proposed} \cite{watanabe2018hate}, intentional repetitions \cite{xiang2012detecting} \cite{hemalatha2012preprocessing} \cite{montani2018tuwienkbs} \cite{rother2018ulmfit} (e.g. ``hhheeeey"), contractions of words \cite{smailovic2013predictive} \cite{hemalatha2012preprocessing}, and spelling mistakes. Although those new word formations do not appear in the dictionary, they are ``intuitive and popular in social media" \cite{hu2012text}. \newline
On Twitter, it is custom to mention other users within a Tweet by adding ``@"+username \cite{xiang2012detecting} \cite{montani2018tuwienkbs} \cite{watanabe2018hate} \cite{rother2018ulmfit}, retweeting (i.e. answering to) a Tweet \cite{xiang2012detecting} \cite{hemalatha2012preprocessing}, and summarizing a Tweet's topic with ``\#"+topic \cite{xiang2012detecting} \cite{watanabe2018hate}. \newline
Other problems in text mining are the handling of stop words \cite{xiang2012detecting} \cite{ghorai2016information} \cite{gupta2018proposed}, language detection \cite{xiang2012detecting}, punctuation \cite{ghorai2016information} \cite{hemalatha2012preprocessing} \cite{montani2018tuwienkbs}, negation \cite{watanabe2018hate}, and case folding \cite{ghorai2016information} \cite{gupta2018proposed} \cite{rother2018ulmfit}.\newline
Researchers have developed different strategies for preprocessing Tweets. One possible approach is to simply remove URLs, username, hashtags, emoticons, stop words, or punctuation \cite{xiang2012detecting} \cite{ghorai2016information} \cite{hemalatha2012preprocessing} \cite{montani2018tuwienkbs} \cite{gupta2018proposed} \cite{watanabe2018hate}. A reason to eliminate those tokens can be that they assumably do not hold information relevant to the classification goal \cite{hemalatha2012preprocessing}. Words that only exist for syntactic reasons (this concerns primarily stop words) can be omitted when focussing on sentiment or other semantic characteristics \cite{ghorai2016information}. Mentions of other users are likewise not informative for sentiment analysis and are often removed from the texts \cite{xiang2012detecting} \cite{watanabe2018hate}. Depending on the dataset size, normalising the texts strongly by removing punctuation and emojis, as well as lowercasing the texts, can decrease the vocabulary size \cite{ghorai2016information}. Especially on Twitter with its restricted text size, users tend to use shortened URLs. Short URLs have a concise, but often cryptic form, and redirect to the website with the original, long URL. While website links can encode some information on a topic, this information is lost when using a shortened URL. Removing the shortened URLs without replacement can be a step in preprocessing Tweets \cite{xiang2012detecting}.\newline
Rather than removing tokens, they can also be replaced by a signifier token, e.g. a complete link by ``$<<<$hyperlink$>>>$" \cite{hovelmann2017fasttext}. In Tweets, such signifier tokens are used for mentions of usernames \cite{smailovic2013predictive} \cite{hovelmann2017fasttext} \cite{rother2018ulmfit}, URLs \cite{smailovic2013predictive} \cite{hovelmann2017fasttext} \cite{rother2018ulmfit}, smilies \cite{hovelmann2017fasttext} or negations \cite{smailovic2013predictive}. Using signifier tokens eliminates some information, i.e. which user was mentioned or which website was linked, but retains the information that a mention or link exists. Tokens can also be grouped by using signifier tokens, i.e. tokens with similar content are summarised with a single token. \cite{hovelmann2017fasttext} uses this technique to group smilies with similar sentiment and Twitter usernames related to the same company.\newline 
Case folding is often addressed by converting Tweets to lower case \cite{ghorai2016information} \cite{hovelmann2017fasttext} \cite{gupta2018proposed}.\medskip\newline
The following preprocessing steps are taken in chronological order:
\begin{enumerate}
	\item Conversion of all texts to lower cases
	\item Replacement of URLs by a dummy URL (``URL")
	\item Replacement of referenced user names and handles by a dummy handle (``USERNAME")
	\item This dataset encodes emojis in unicode decimal codes, e.g. ``\&\#128512;" for a grinning face. In order to keep the information contained in emojis, each emoji is replaced by its textual description (upper cased and without whitespaces to ensure unity for tokenizing)\footnote{https://www.quackit.com/character\_sets/emoji/}.
	\item Resolving contractions such as ``we're" or ``don't" by replacing contractions with their long version\footnote{https://en.wikipedia.org/wiki/Wikipedia:List\_of\_English\_contractions}.
	\item This dataset uses a few signifiers such as ``english translation" to mark a Tweet that has been translated to English, or ``rt" to mark a Retweet (i.e. a response to a previous Tweet). Since those information have been added retrospectively, we discard them here and delete the signifiers from the texts.
	\item Replacement of all characters that are non-alphabetic and not a hashtag by a whitespace
	\item Replacement of more than one subsequent whitespace by a single whitespace
	\item Tokenization on whitespaces
\end{enumerate}
After training the classifiers, the URL and username tokens are replaced by a more readable version (``http://website.com/website" and ``@username", respectively) to make it easier for participants of the user study to envision themselves in the scenario of a social media administrator reading real-world Tweets. Replacing the tokens by their original URLs and usernames would give the participants more information than the classifiers had; we therefore chose to use a dummy URL and username.\medskip\newline

Following the preprocessing steps, the following Tweet is processed from its original form:
\medskip \hrule \medskip
\begin{verbatim}
	"@WBUR: A smuggler explains how he helped fighters along the \end{verbatim}\begin{verbatim}"Jihadi Highway": http://t.co/UX4anxeAwd"
\end{verbatim}
\medskip \hrule \medskip
into a cleaned version:
\medskip \hrule \medskip
\begin{verbatim}
@username a smuggler explains how he helped fighters along the \end{verbatim}\begin{verbatim}jihadi highway http://website.com/website
\end{verbatim}
\medskip \hrule \medskip





\subsection{Classifier}
Intro

\paragraph{Good System}
L2X 

\paragraph{Medium System}
Logistic Regression with binary (1 / -1) coefficients

\paragraph{Bad System}
Inverse L2X



\subsection{Explanations}
reference to section \ref{subsubsec:explanation_systems}: explanations for text via highlighting of important features (here: words). 

\paragraph{Good System}
L2X mutual information

\paragraph{Medium System}
randomly choosing k words from the words with positive (offensive) or negative (not offensive) class

\paragraph{Bad System}
Inverse good system



\subsection{Graphical User Interface}
\begin{itemize}
	\item the ``environment": administration tool
	\item visualisation of the decisions
	\item visualisation of the explanations (why changing text colour, why those two colours, why not gradient)
\end{itemize}

\begin{figure} [H]
	\centering
	\includegraphics[width=0.8\textwidth]{img/administrationTool.JPG}\\
	\caption{Screenshot of the ``Administration Tool" to support the scenario of a social media administrator}
	\label{fig:admin_tool}
\end{figure}
\begin{figure} [H]
	\centering
	\includegraphics[width=0.8\textwidth]{img/pg_2_12.PNG}\\
	\caption{Screenshot of the ``Administration Tool" showing an offensive Tweet with explanation for its decision}
	\label{fig:admin_tool_offensive}
\end{figure}
\begin{figure} [H]
	\centering
	\includegraphics[width=0.8\textwidth]{img/pg_2_0.PNG}\\
	\caption{Screenshot of the ``Administration Tool" showing a non-offensive Tweet with explanation for its decision}
	\label{fig:admin_tool_not_offensive}
\end{figure}



\subsection{Subset Sampling}
For evaluating the different system-explanation conditions, users have to experience the system. However, it is not feasible to present them with the complete testeset, since it has a size of 1665 Tweets. Consequently, a subset of Tweets needs to be drawn from the testset, with a size that a human observer can understand and process within the time frame of a user study.\newline
We furthermore aim to find 10 suitable subsets and assign participants randomly to one of the subsets, in order to reduce possible side effects from biases specific to single Tweets.\newline
There are several requirements for the subsamples, originating from the conflict of reducing the sample for a human observer, yet still yielding a good representation of the testset and classifier:\newline
\begin{itemize}
	\item A class balance of the true labels similar to the testset, 
	\item a balance of correctly to incorrectly classified data points similar to the classifier's performance on the complete testset, 
	\item no overlap of Tweets within the set of 10 subsets,
	\item a feature distribution as close to the feature distribution in the complete testset.
\end{itemize}
We set the subsample size to 15 Tweets, which is enough to show accuracies to the first decimal place, yet assumably not too much to process for an observer in a user study.\newline
To create a subset, 15 data points are randomly drawn from the testset. \newline
First, the class balance of the subset is calculated. The difference to the class balance of the whole testset needs to be smaller than 0.1.\newline
Additionally, for each classifier in the user study, the prediction accuracy on the subset is compared to the prediction accuracy on the complete testset. If, for all classifiers, the difference is smaller than 0.1, the next check is performed.\newline
To ensure the uniqueness of the subsets, the randomly drawn Tweets are compared with the content of previously found subsets. The subset is only accepted if none of the contained Tweets appear in any previously found subset.\newline
In the last step, the feature distribution of the subset is tested against the features of the complete testset using the \textit{Kullback-Leibler Divergence} (KLD) metric. As the focus is directed towards the explanations (i.e. the highlighted words within a Tweet), only the explanations are used to examine the feature distribution. First, the feature distribution of the complete testset is calculated by constructing a word vector with tuples of words and their respective word counts. The word counts are divided by the total amount of words in the set, such that the sum of regularised counts equals 1. Next, a copy of the word vector is used to count and regularise the word frequencies in the subset. The result are two comparable vectors, yet the vector of the subset is very likely to contain zero counts for words that appear in the complete set but were never selected as explanation in the subset. Since the KLD uses the logarithm, it is undefined for zero counts. We use Laplace smoothing with k=1 to handle zero counts. For each classifier, the KLD is calculated and summed to a total divergence score for the subset.\newline
We generate a quantity of 100 such subsets and order them by their KLD sum. The 10 subsets with the smallest score are chosen as the final set of subsets.\newline




\subsection{Explanation Evaluation}
\label{subsec:expleval}
experiments to validate whether the explanations we generate are actually ``good" explanations and those generated with the random method are actually ``bad" explanations.

\subsubsection{Evaluation 1: Accuracy on Reduced Texts}

\subsubsection{Evaluation 2: Ability to Reproduce Classifications}



\subsection{Subset Evaluation}
same as previous section, but with the filtered subsets, basically. And on ``perfect" classifier.
\section{User Study: Trust Evaluation}
In the previous section, we discussed three systems with different accuracy levels and three types of explanations. Similar to the experiment discussed in \cite{langer1978mindlessness}, we have built a system offering (1) no explanation for its decision, (2) a placebic explanation (non-informative) for its decision, and (3) an informative (i.e. truthful) information for its decision. In this section we present the user study in which we investigated the influence of model accuracy and explanation fidelity on user trust. We use two approaches to measure user trust: an explicit measure based on a questionnaire and a proxy that measures trust via the willingness to accept and adapt to the system's recommendations.


\subsection{Method}

\paragraph{Participants}
In total, 327 participants took part in the main user study with an average age of 29.4 years (SD = 8.8), a gender balance of 56\% (males) to 43\% (females) and two participants reporting the third gender. 87\% of the participants were recruited via the paid science crowdsourcing platform ``Prolific"\footnote{https://prolific.ac}, while 36 participants enlisted on ``SurveyCircle"\footnote{https://www.surveycircle.com}, an unpaid participant recruitment platform based on mutuality.\newline
On both platforms, individuals younger than 18 years were excluded to participate for reasons of consent by a major. The use case scenario included reading and understanding real-life Tweets with slang words, grammatical and literal errors. The platforms therefore screened for people being fluent in English. 57\% self-assessed their level of English to be equivalent to a native speaker, 23\% as advanced (C1 on the Common European Framework of Reference for Language scale \cite{council2001common}), 14\% as upper-intermediate (B2), and 5\% as lower than that. All participants claimed to be ``fluent" in English. The study questionnaire included an attention check question, asking the participants to answer ``completely disagree" in between the trust questionnaire items assessed on a 5-point Likert scale. Data from participants who failed to answer the attention check correctly was excluded from the analysis. Furthermore, only complete responses were used in the analysis, i.e. data from participants who reached the last page of the survey. The exclusion criteria invalidated 41 data points, resulting in 286 valid cases.\newline
All participants recruited on the paid platform ``Prolific" received a compensation of 1.40 GBP (1.60 EUR) for an estimated completion time of 12 minutes. Participants from ``SurveyCircle" received a reward of 4.4 Study Points.

\paragraph{Apparatus}
The user study was set up as an online study, the study could therefore be taken at a self-chosen location on private devices. Participants were asked to completed the survey on a notebook, desktop computer or tablet. For consistency with the use case scenario, screenshots of a fictive social media management platform showed the input texts, decisions and explanations. The screenshots had a ratio of 900px (width) to 253px (height). To ensure that improper scaling of the screenshots did not influence the participants' perception, devices with small screens (e.g. smartphones and other mobile devices) were excluded. However, which device participants finally used could not be verified. No further requirements were made regarding the equipment of the participant's device.

\paragraph{Procedure}
On both platforms, the participants receive a link to the survey. As soon as the participant has opened the survey URL, the survey starts. The survey consists of the following content:
\begin{enumerate}
	\item Introduction \& consent form
	\item \textit{Scenario 1}: Social media administrator and manual offensive language detection
	\item \textit{Tweet block 1}: 15 Tweets for classification, on individual pages (no system)
	\item \textit{Scenario 2}: Introduction to automatic decision system supporting the task
	\item \textit{Tweet block 2}: Repetition of 15 Tweets for classification, on individual pages (with system)
	\item Perceived understanding \& trust questionnaire
	\item Demographics
	\item Outroduction \& crowdsourcing completion codes
\end{enumerate}
In general, the study contains three blocks plus an introduction and outroduction section. The first block treats a scenario in which the participant plays the role of a ``social media administrator" of a company with a young target group (15-20 years old). The task of the social media administrator is to identify content with offensive language in order to block such comments or Tweets. The next 15 pages of the survey contain one Tweet each, shown on a screenshot of a management tool, and ask the participant to classify the text as offensive or not offensive as shown in figure \ref{fig:survey_tools1}. The order in which the Tweets are shown is randomised for each participant. There are 10 different sets of Tweets available (without overlap), to avoid effects from the specific wording or topics in the small set of 15 Tweets. At the start of the survey, each participant is randomly assigned to one Tweet set by the system.\newline
\begin{figure} [H]
	\centering
	\includegraphics[width=0.8\textwidth]{img/neu_5_13.JPG}\\
	\caption{Screenshot of the fictive management tool without the automatic decision system}
	\label{fig:survey_tools1}
\end{figure}
The second block introduces the automatic decision system (see figure \ref{fig:survey_tools2}). The participant is again asked to classify 15 ``very similar" Tweets, which are, in fact, identical to the ones shown in the first block. This particular formulation aims to liberate the participants from the urge to classify each text with exactly the same label as in the first block. The ordering of the Tweets is random and hence very likely to be different from the ordering of the first block. In total, 9 conditions exist: three systems (classifier with 0.95, 0.75, and 0.05 accuracy) with three explanation types (informative, placebic, no explanation) each. Each participant has one condition assigned at the beginning of the survey, such that there is an equal distribution of conditions in finished questionnaires. \newline
\begin{figure} [H]
	\centering
	\includegraphics[width=0.8\textwidth]{img/pg_5_13.PNG}
	\caption{Screenshot of the fictive management tool with the automatic decision system}
	\label{fig:survey_tools2}
\end{figure}
Finally, the last block contains three questions regarding perceived understanding, 19 items measuring the user's trust including an attention check, and 5 demographic questions (gender, age, country, ethnicity, English language level).\newline
A detailed list of all questions used in the survey can be found in appendix {\color{red}X}.

\paragraph{Design \& Analysis}
The between-subject setup described in the previous paragraph was tested in a pilot study with 11 participants. The participants were recruited via ``Prolific" and received a compensation of 2.00 GBP (2.28 EUR). They completed the study in ``pretest" mode, which shows an additional comment box at the bottom of each survey page.\newline
The main study was set up as a quantitative study without open questions or free text input. Basic frequency analysis was used for the demographic items in order to understand the background of the participants. Three topics were investigated in a statistical manner: perceived understanding (3 items), self-reported trust (19 items), and observed trust via proxy. For the first two, a 5-point Likert scale was employed.\newline
A \textit{Perceived understanding} score was calculated for each participant by taking the mean of the ratings for all three items in the questionnaire. The trust questionnaire used to measure \textit{self-reported trust} contains 14 positive items and 5 inverse items. A single mean score was calculated by taking the average over the positive items and the maximum rating minus the mean of the inverse items. As a second trust measure, \textit{observed trust} was investigated via the proxy of willingness to follow a system's recommendation. The survey contained one block of manual classification without the system, and a second round with the information provided by the automated decision system. In each block, participants classified the same set of Tweets. We can therefore determine how often a participant switched his or her classification out of 15 possible cases and how often the change was made in agreement with the classifier's prediction but against the truth. Since the three classifiers offered different amount of opportunities to change with the classifier's prediction away from the truth (maximum 14 cases for the bad classifier as opposed to maximum 1 case for the very good classifier), the proxy measure is calculated and normalised as follows for each participant:
\[ \frac{changes\_towards\_prediction\_against\_truth}{opportunities\_for\_change\_against\_truth} \]
Cases in which the very good classifier did not make any misclassification (hence no opportunity for the user to change in favour of the classifier and in contradiction to the truth) were excluded, because no valid conclusion can be drawn from these cases. 42 cases occurring in the conditions with the very good classifier had to be excluded due to this issue.\newline
%14,17,23
%30,32,34
The goal of the statistical analysis for all three topics (perceived understanding, self-reported trust, observed trust via proxy) is to identify differences between different conditions. Not all samples were normally distributed, which we investigated with the Shapiro-Wilk test \footnote{https://docs.scipy.org/doc/scipy/reference/generated/scipy.stats.shapiro.html} for normality from the SciPy library). We therefore used the Mann-Whitney U test to compare two samples, since it does not assume normal distribution nor equal sample sizes or variances. For sample sizes above 20 data points, we employed SciPy's approximation\footnote{https://docs.scipy.org/doc/scipy/reference/generated/scipy.stats.mannwhitneyu.html} of the Mann-Whitney U test. For smaller sample sizes - only occurring in the observed trust via proxy scores where data points had to be excluded -, we used the exact implementation\footnote{https://mail.python.org/pipermail/scipy-dev/2015-March/020475.html} of the Mann-Whitney U test as described in \cite{cheung1997mann}.


\subsection{Results}
The following section presents the results of the user study. We examined perceived understanding, self-reported trust and an implicit trust measure via the willingness to follow a classifier's recommendation. For each topic, we give the mean score, standard deviation, as well as a comparison of all conditions in a 9x9 matrix.\newline
The matrices show each condition checked for significant difference with every other condition. The colour scale is a visualisation of the p-value: Insignificant p-values, i.e. values above the critical threshold of 0.05, are coloured in dark blue, while significant p-values are presented with colours from blue over green to light yellow. P-values marked as ``0" are too small (below 0.001) to be displayed correctly in the matrix.\newline 

\paragraph{Perceived Understanding}
As figure \ref{fig:results_understanding} shows, users of the system with a very good classifier and no explanation report the highest perceived understanding. For the very good and the medium classifier, giving no explanations for the decisions leads to a higher perceived understanding than delivering placebic, i.e. random, explanations. In general, users have more confidence in their understanding of the system for the very good and medium classifiers as compared to the bad classifier. One condition, however, does not lead to significantly higher scores than the bad classifier: for the medium classifier with random explanations, users reported the same understanding as for the bad classifier with no explanations. Concerning the bad classifier, giving a truthful explanation for the decision leads to the lowest perceived understanding.
\begin{figure}[H]
	\begin{subfigure}[b]{0.3\textwidth}
		\raisebox{80pt}{\resizebox{\textwidth}{!}{
				\begin{tabular}{lrr}
					\textbf{Condition} & \textbf{Mean} & \textbf{SD} \\ \midrule
					super-good & 11.833 & 2.746 \\
					super-rand & 11.188 & 2.579 \\
					super-no & 12.441 & 2.103 \\
					medium-good & 11.455 & 2.475 \\
					medium-rand & 8.833 & 2.888 \\
					medium-no & 11.100 & 2.071 \\
					bad-good & 7.395 & 3.602 \\
					bad-rand & 7.500 & 3.413 \\
					bad-no & 8.833 & 3.455 \\ \bottomrule
		\end{tabular}}}
		\caption{Mean and standard deviation per condition}
		\label{tab:results_table_understanding}
	\end{subfigure}
	\begin{subfigure}[b]{0.65\textwidth}
		\includegraphics[width=\textwidth]{img/results_matrix_understanding2.JPG}
		\caption{Significance matrix with p-values per condition}
		\label{fig:results_matrix_understanding}
	\end{subfigure}
	\caption{Results for perceived understanding scores}
	\label{fig:results_understanding}
\end{figure}


\paragraph{Trust Questionnaire}
The self-reported trust scores show similar results as the perceived understanding: Besides the medium classifier with random explanations, all systems lead to significantly more trust than the systems employing the bad classifier. The explanations do not play a role regarding user's trust when the bad classifier is used. Looking at the medium classifier, the random explanation leads to a lower trust score than no explanation and a good explanation, with no difference between the latter two. The most trust is evoked by the very good classifier without explanations, significantly more than for any other condition. There is no significant difference between the very good classifier with explanations and the medium classifier with meaningful explanation. For both the bad classifier and the very good classifier, the condition without any explanation again led to the highest scores within the same classifiers. The detailed results are presented in figure \ref{fig:results_trust}.
\begin{figure}[H]
	\begin{subfigure}[b]{0.3\textwidth}
		 \raisebox{80pt}{\resizebox{\textwidth}{!}{
		\begin{tabular}{lrr}
			\textbf{Condition} & \textbf{Mean} & \textbf{SD} \\ \midrule
			super-good & 50.967 & 7.600 \\
			super-rand & 50.906 & 9.156 \\
			super-no & 56.912 & 9.721 \\
			medium-good & 50.030 & 9.150 \\
			medium-rand & 42.000 & 9.671 \\
			medium-no & 49.967 & 8.712 \\
			bad-good & 36.421 & 8.129 \\
			bad-rand & 37.067 & 7.659 \\
			bad-no & 38.333 & 10.381 \\ \bottomrule
		\end{tabular}}}
		\caption{Mean and standard deviation per condition}
		\label{tab:results_table_trust}
	\end{subfigure}
	\begin{subfigure}[b]{0.65\textwidth}
		\includegraphics[width=\textwidth]{img/results_matrix_trust2.JPG}
		\caption{Significance matrix with p-values per condition}
		\label{fig:results_matrix_trust}
	\end{subfigure}
	\caption{Results for self-reported trust scores}
	\label{fig:results_trust}
\end{figure}



\paragraph{Observed Trust via Proxy}
The second trust measure uses a proxy to determine the trust a user puts into a system: the willingness to follow a system's recommendation, in this case the decision about offensiveness and non-offensiveness. Figure \ref{fig:results_proxy_away} shows the results of analysing the user's willingness to change a classification to match the system's decision while contradicting the truth. As a comparison, figure \ref{fig:results_proxy_towards} deals with changes in classification that were made in favour of both the system and the truth.\newline
The highest changing rate in favour of the system but against the true label was detected for users of the very good classifier with a meaningful explanation, but also the highest variance. Users were significantly more likely to adapt the system's faulty decision when confronted with the very good system with random and no explanations than the users of any system with the bad classifier. The same holds true for users of the medium classifier without explanations.\newline
\begin{figure}[H]
	\begin{subfigure}[b]{0.3\textwidth}
		\raisebox{80pt}{\resizebox{\textwidth}{!}{
				\begin{tabular}{lrr}
					\textbf{Condition} & \textbf{Mean} & \textbf{SD} \\ \midrule
					super-good & 0.286 & 0.452 \\
					super-rand & 0.118 & 0.322 \\
					super-no & 0.043 & 0.204 \\
					medium-good & 0.088 & 0.189 \\
					medium-rand & 0.083 & 0.158 \\
					medium-no & 0.075 & 0.183 \\
					bad-good & 0.050 & 0.066 \\
					bad-rand & 0.054 & 0.067 \\
					bad-no & 0.073 & 0.093 \\ \bottomrule
		\end{tabular}}}
		\caption{Mean and standard deviation per condition}
		\label{tab:results_table_proxy_away}
	\end{subfigure}
	\begin{subfigure}[b]{0.65\textwidth}
		\includegraphics[width=\textwidth]{img/results_matrix_proxy_away2.JPG}
		\caption{Significance matrix with p-values per condition}
		\label{fig:results_matrix_proxy_away}
	\end{subfigure}
	\caption{Results for proxy measure of trust via willingness to accept the system's prediction changing manual label away from actual truth}
	\label{fig:results_proxy_away}
\end{figure}
Looking at the changes made towards the truth in agreement with the classifiers, no significant differences are noted between any condition with the very good and medium classifier. The same holds true for the bad classifier. The very good and medium classifiers, however, evoked significantly more changes towards the truth than the bad classifier with explanations. The standard deviations of the conditions using the bad classifier are rather high as compared to any other condition.\newline
One condition is exceptional in this analysis: Although the bad classifier without explanation has the highest mean score (i.e. changes towards the truth when the classifier made a correct prediction), the score is not significantly different from the bad classifier with a good and random explanation. The variance of all three systems (bad-no, bad-random, bad-good) are very high as compared to the variances of the other systems. The score deviates, however, from the results of the very good and medium classifier, which have lower mean scores but lower variances. The difference in variance is important to note when comparing the relatively high mean score of the bad classifier without explanation to the conditions with the very good and medium classifiers.
\begin{figure}[H]
	\begin{subfigure}[b]{0.3\textwidth}
		\raisebox{80pt}{\resizebox{\textwidth}{!}{
				\begin{tabular}{lrr}
					\textbf{Condition} & \textbf{Mean} & \textbf{SD} \\ \midrule
					super-good & 0.073 & 0.073 \\
					super-rand & 0.047 & 0.064 \\
					super-no & 0.063 & 0.093 \\
					medium-good & 0.058 & 0.087 \\
					medium-rand & 0.049 & 0.070 \\
					medium-no & 0.058 & 0.066 \\
					bad-good & 0.048 & 0.213 \\
					bad-rand & 0.045 & 0.208 \\
					bad-no & 0.091 & 0.287 \\ \bottomrule
		\end{tabular}}}
		\caption{Mean and standard deviation per condition}
		\label{tab:results_table_proxy_towards}
	\end{subfigure}
	\begin{subfigure}[b]{0.65\textwidth}
		\includegraphics[width=\textwidth]{img/results_matrix_proxy_towards2.JPG}
		\caption{Significance matrix with p-values per condition}
		\label{fig:results_matrix_proxy_towards}
	\end{subfigure}
	\caption{Results for proxy measure of trust via willingness to accept the system's prediction changing the manual label towards the truth}
	\label{fig:results_proxy_towards}
\end{figure}



\subsection{Discussion}
\section{Discussion}
%-------------------------------------------------------------------------
% RQ 1: effect of accuracy on trust
Our results suggest that \textit{both perceived understanding and trust are positively related to the classifier's accuracy in general} (\textbf{RQ 1}). The higher the accuracy, the higher the trust in the system. The strongest evidence is found in the reactions to the classifiers without explanations (\textit{super-no}, \textit{medium-no}, \textit{bad-no}). The self-reported trust was the highest for \textit{super-no} and the lowest for \textit{bad-no}, with all scores differing significantly from each other. The same can be observed even when adding the bad explanations: The very good classifier still has significantly higher trust and perceived understanding scores than the other classifiers, with the bad classifier again having the lowest scores. The good explanation, however, influences the trust and perceived understanding differently. Here, the bad classifier still receives significantly worse trust and understanding scores than the other two, yet there is no difference anymore in the scores for the very good and medium classifiers. \newline
An explanation for the similarity of both the trust score and the perceived understanding score for super-good and medium-good could be the persuasiveness of a good explanation. The difference in trust and perceived understanding, that we see between the very good and medium classifier in the condition without explanation, could be compensated by convincing the user of the classifier's trustworthiness through a good explanation.\newline
It seems intuitive to have higher trust in a system that leads to fewer deception, which has also been described in \cite{glass2008toward} with the ``expectation mismatch" (see section \ref{subsubsec:trust_factors}). A classifier with high accuracy effectively leads to fewer disappointed expectations, which in turn does not decrease the trust. Furthermore, the set size of 15 Tweets seems to be enough for users to develop an intuition about the classifier's accuracy. \medskip \newline
%-----------------------------------------------------------------------
% RQ 2: effects of explanation types on trust
In the copy machine experiment by \cite{langer1978mindlessness}, only the pure presence of an explanation was enough to make people comply with a request resulting in a short waiting time. On the basis of that experiment, we designed three explanations similar to the setup in \cite{langer1978mindlessness}: No explanation, placebic explanation, and a meaningful explanation for the classifier's behaviour. Similar to the results of the copy machine experiment, we expected to see no difference between trust scores of the meaningful and placebic explanation but a difference between the two explanations and the no explanation settings. The results, however, show a mixed answer.\newline
For the \textit{very good classifier}, the meaningful and placebic explanation indeed led to the same trust score. Other than expected, the no explanation condition showed the best results. The classifier performed at an accuracy of 0.95, which resulted in 44\% of the cases in a perfect classification rate within the small subset of 15 Tweets. A possible explanation for the good trust score in the no explanation condition could be the conservation of a perfect image throughout the 15 Tweets. The classifier makes (almost) no mistakes and does not offer any information that could lead to doubts about the classifier's abilities. Both displayed explanation types would then have a disadvantage over the no explanation condition: The good explanations are not necessarily meaningful to a human, as they are based on statistical information rather than semantics or intentions. The placebic explanation is generated at random, which likewise holds potential for doubts and incomprehension. A similar guess was ventured in \cite{cramer2008effects}, who suspected that more knowledge about system boundaries and unfulfilled preferences leads to a decrease in trust (see section \ref{subsubsec:trust_factors}). The opposite can be observed in the proxy measure for trust, i.e. the changes in labelling that a user made towards the classifier's decision but away from the truth. Here, the no explanation condition led to significantly fewer changes away from the truth as compared to the two conditions giving any type of explanation, which is in turn consistent with the results in the copy machine experiment. It is important to note that the copy machine experiment only worked while people are in a ``mindless" state, i.e. an inattentive state of mind. It is possible that users did not in particular pay attention to their trust towards the system during the classification task, but actively reflected on their relationship with the system during the self-report of trust. Being in a mindless state during the proxy measurement while being mindful during the trust questionnaire would explain the conflicting results of both measures.\newline
The results for the \textit{medium classifier} differ from those of the very good classifier. The two conditions with explanations have significantly different trust scores. The placebic explanation has the lowest score, while the meaningful explanation is ranked at the same trust level as the system without explanation. The classifier delivers faulty classifications in three to four cases out of 15, which presumably raises doubts about the system. The negative effect of placebic explanations could therefore be worsened. The ``expectation mismatch" is then twofold, with the wrong classification on the one side and the useless explanation on the other side. With the good explanations, the users receive some information about the underlying reasons for a misclassification. Even if not all information of the explanation is meaningful to a human, it delivers hints to the system's function and malfunction, possibly raising overall trust.\newline
The \textit{bad classifier} did not show evidence of diverting trust scores for any of the three explanation types. The same homogeneity is found in the results of the proxy measurement of trust, for both the changes away from the truth and towards. The trust scores were significantly lower than any other condition, with one exemption. The bad classifier without explanation had comparable trust ratings as the medium classifier with random explanations. Since there is a significant distance between the scores of the medium classifier with random explanations and other conditions of the medium classifier, we conclude that it is due to a property of the medium-random system rather than a phenomenon of the bad classifier. The evidence suggests that users are not fooled by a bad classifier and do not trust it, no matter the explanation given. A plausible assumption could have been that users trust a bad classifier as it is predictable, but do not use it as a basis for their decisions because it is not accurate. This distinction, however, is not found in the results.\newline
Overall, we found evidence that the accuracy of a classifier is more important for trust than the explanation. The explanations did not make a difference for the very good classifier nor the very bad classifier (\textbf{RQ 2}). The case of the medium classifier is an interesting one, as we found an influence of the explanations on user trust here. It would be interesting to investigate the relationship of users with the medium classifier in more detail in future research. The findings also show that an evaluation of explanations in xAI should not only be made for extreme cases, but also consider the - supposedly more realistic - cases on the whole spectrum between the extremes. \medskip \newline
%--------------------------------------------------------------------------------
% RQ 3: effect of explanation types on perceived understanding
One of the factors contributing to trust is \textit{perceived understanding}. Our findings show a negative influence of meaningless information on perceived understanding (\textbf{RQ 3}). For both the medium classifier and the very good classifier, perceived understanding was the worst when delivering placebic explanations. Although actual understanding is arguably different when comparing a case without any explanation and one with good, meaningful information, the perception of knowledge about both cases is equal here. A mechanism similar to ``expectation mismatch" could be in place for perception of knowledge. While building the mental model of the classifiers, no conflicting information have to be consolidated for the good explanation and the no explanation cases. Being confronted with random and therefore meaningless explanations forces the user to unite conflicting information in the mental model. The more conflicts appear, the lower the confidence in the mental model.\newline
For the bad classifier, perceived understanding ratings are significantly lower as for other classifiers (except for medium-random, which has a low rating as well). However, the system delivering good explanations for the faulty behaviour receives a significantly higher score than no explanation and placebic explanation cases. The positive effect of high accuracy does not hold here because the classifier performs badly on the task. Yet, as the explanations give more information about the inner workings of the classifier, it seems intuitive to evoke more confidence of understanding in this case. \medskip \newline
%--------------------------------------------------------------------------------
% Further discussions
In this research, we used a computational evaluation to validate the fidelity of the automatically generated explanations. Although the ``meaningful" explanations demonstrably represent the features that are decisive for the classification, they are not necessarily meaningful to a human observer. We showed in section \ref{subsec:expleval} that the selected words in the texts were enough to reconstruct the behaviour of the classifiers and can therefore serve as a basis for decision. Whether the selection of words is enough for humans to judge is up to discussion. Further research is necessary to determine the actual ``meaningfulness" of the generated explanations for humans.\newline
The proxy measurement of trust via changes in classification between the first and second block of Tweets showed ambiguous results with high variance. For future projects dealing with trust in computer systems, it could be useful to measure trust not only via a questionnaire that requires reflection abilities and active processing of the relationship between user and system. Using a trust measure that can be determined without the participant knowing could serve as an additional view on the practical implications of trust.

%--------------------------------------------------------------------------------
% Further discussions
\paragraph{Future work}
amount of explanatory content (like \cite{ribeiro2018anchors})

how trust is built up over time:
Whether users start with a high trust level and decrease the trust with every mismatch, or have a basic trust level that is increased with expectation matches and decreased with every mismatch remains to be examined in future research. The results also do not deliver information about the proportionality of accuracy and the trust level, which could be a topic of future research as well. 


%--------------------------------------------------------------------------------
% Further discussions
\paragraph{Societal impact}
a warning sign that we need to avoid inappropriate trust in computer systems, but:
accuracy weighs strongest --> people not easy to trick
high-level study, should be tested again with actual discrimination (not easy to get the data, though)


\section{Conclusion}
This paper presents empirical evidence for the influence of accuracy and explainability on user trust. First, we generated truthful minimal explanations and nonsensical explanations for three systems with different performance levels. We then validated the explanations' fidelity by reducing sample points to input features used for the explanations, and re-classifying those sample points a second time with the same classifiers. We built a use scenario and tested the explanations in a user study. The users' trust was measured with a self-reporting questionnaire and a proxy measure based on observations of how the participants' classifications were influenced by seeing the explanations.\newline
Our findings show that users have the most trust in systems without explanations, i.e. minimum explanations can potentially harm, but not improve user trust. We argue that the act of reconciling conflicting information of the mental model and the given explanations counts as a deceptive experience and therefore affects the user's trust negatively. If an explanation is added to a system (e.g. for increasing user's understanding of the system), its truthfulness is crucial for user trust. We saw that for systems with a medium accuracy (0.76), a truthful explanation does not harm user trust, while a nonsensical explanation decreases trust. Overall, the systems' accuracy levels were most decisive for user trust: the higher the accuracy, the higher the user's trust.\newline
Further research with more rich explanations and a detailed investigation of trust factors is needed to examine potential positive effects of explanations on user trust. The development of trust over time should also be researched in the future, to give valuable directions to xAI practitioners implementing explanations in productive systems.



\bibliographystyle{abbrv}
\bibliography{mybib}

\addtocontents{toc}{\protect\setcounter{tocdepth}{0}}
\appendix
\section{Appendix}

The survey of the user study contains the following elements:
\begin{enumerate}
	\item Introduction \& consent form
	\item \textit{Scenario 1}: Social media administrator and manual offensive language detection
	\item \textit{Tweet block 1}: 15 Tweets for classification, on individual pages (no system)
	\item \textit{Scenario 2}: Introduction to automatic decision system supporting the task
	\item \textit{Tweet block 2}: Repetition of 15 Tweets for classification, on individual pages (with system)
	\item Perceived understanding \& trust questionnaire
	\item Demographics
	\item Outroduction \& crowdsourcing completion codes
\end{enumerate}
This section lists the scenario formulation used in the study, as well as the questionnaires for perceived understanding, trust, and demographics.

\subsection{Scenario 1: Scene}
For answering the upcoming questions, \textbf{please read the following scenario carefully} before continuing to the next page. \medskip \newline
Imagine you work for a youth TV channel that targets \textbf{people at the age of 15-20 years}.
You are the company's \textbf{social media administrator}.\newline 
Your responsibilities are:
\begin{enumerate}
	\item Upload new content on various platforms, e.g. Facebook, Twitter, Instagram
	\item Answer messages
	\item Moderate discussions and comments
	\item Ensure that the language of comments is appropriate: delete comments with offensive language and block users
\end{enumerate}
To support your work, you are given the ``\textbf{Administration Tool}". The ``Administration Tool" is a program that notifies you of new messages, lets you upload content to your social media channels, and alerts you in case of an inappropriate comment.
\begin{figure} [H]
	\centering
	\includegraphics[width=0.8\textwidth]{img/administrationTool.JPG}
	\label{fig:app_admin_tool}
\end{figure}
\noindent Today, your manager asked you to review Tweets that may or may not contain \textbf{inappropriate, offensive language}.\newline 
To protect the \textbf{children and young adults} visiting your page, your company has strict rules on what ``offensive" means:\newline 
\begin{enumerate}
\item Containing \textbf{hateful} language: any comment that disparages a person or a group on the basis of some characteristic such as race, colour, ethnicity, gender, sexual orientation, nationality, and religion
\item Containing \textbf{pornographic} language: explicit sexual subject matter for the purposes of sexual arousal and erotic satisfaction
\item Containing \textbf{vulgar} language: coarse and rude expressions, which include explicit and offensive reference to sex or bodily functions
\item Not only single words can be offensive, but also the \textbf{meaning} of a text. A text can be offensive without explicitly mentioning offensive words.
\end{enumerate}
On the next page, you will see 15 screenshots of the ``Administration Tool".\newline
\textbf{Please classify the Tweets on the screenshots using the above-mentioned guidelines into ``offensive" and ``not offensive".}\newline
The Tweets are taken directly from social media and may therefore include \textbf{typos, grammatical mistakes, and slang words}.\newline



\subsection{Tweet Block 1: Without System}
An exemplary screenshot of a survey page in block 1:
\begin{figure} [H]
	\centering
	\includegraphics[width=0.8\textwidth]{img/app_screenshot_no_sys.JPG}
	\label{fig:app_block1}
\end{figure}



\subsection{Scenario 2: Task}
For answering the upcoming questions, \textbf{please read the following scenario carefully} before continuing to the next page. \medskip \newline
The ``Administration Tool" team has developed a system that automatically detects offensive Tweets.
The automatic detection system shows the result of its computation as follows:
\begin{figure} [H]
	\centering
	\includegraphics[width=0.8\textwidth]{img/example_decNotOffensive.JPG}
	\label{fig:app_decision_non_offensive}
\end{figure}
and 
\begin{figure} [H]
	\centering
	\includegraphics[width=0.8\textwidth]{img/example_decOffensive.JPG}
	\label{fig:app_decision_offensive}
\end{figure}
\noindent On the next page, you will see 15 Tweets very similarly to the ones before. This time, the automatic detection system assists you in your task.\newline
\textbf{Please classify the Tweets on the screenshots into ``offensive" and ``not offensive"}.



\subsection{Tweet Block 2: With System}
An exemplary screenshot of a survey page in block 2:
\begin{figure} [H]
	\centering
	\includegraphics[width=0.8\textwidth]{img/app_screenshot_sys.JPG}
	\label{fig:app_block2}
\end{figure}



\subsection{Questionnaire Perceived Understanding}
\textbf{Please evaluate how confident you are in your understanding of the automatic classification system.}
\begin{enumerate}
	\item The decision (``offensive", ``not offensive") the system made for a single Tweet	
	\item The reason why the system made the decision	
	\item The important factors that contributed to the decision
\end{enumerate}
Evaluation on a 5-points Likert scale with labelling of the extrema: ``not confident that I understand" (left) and ``totally confident that I understand" (right)



\subsection{Questionnaire Trust}
\textbf{Please indicate your level of agreement with the following statements.}
\begin{enumerate}
	\item The system is capable of interpreting situations correctly.
	\item The system state was always clear to me.
	\item I already know similar systems.
	\item The system's developers are trustworthy.
	\item One should be careful with unfamiliar automated systems.
	\item The system works reliably.
	\item The system reacts unpredictably.
	\item The system's developers take my well-being seriously.
	\item I trust the system.
	\item A system malfunction is likely.
	\item I was able to understand why things happened.
	\item Please tick ``strongly disagree" to show that you are not a robot.
	\item I rather trust a system than I mistrust it.
	\item The system is capable of taking over complicated tasks.
	\item I can rely on the system.
	\item The system might make sporadic errors.
	\item It's difficult to identify what the system will do next.
	\item I have already used similar systems.
	\item Automated systems generally work well.
	\item I am confident about the system's capabilities.
\end{enumerate}
Evaluation on a 5-points Likert scale with labelling of each point: ``strongly disagree", ``rather disagree", ``neither disagree nor agree", ``rather agree", ``strongly agree".



\subsection{Demographics}
\textbf{What is your age?}
\begin{itemize}
	\item 18-30 
	\item 31-40
	\item 41-50
	\item 50+
\end{itemize}
\textbf{With which gender do you identify most?}
\begin{itemize}
	\item Male
	\item Female
	\item Inter / diverse
\end{itemize}
\textbf{In which culture were you primarily raised?}
\begin{itemize}
	\item Caucasian (e.g. European, North American, Central Asian)
	\item Latino / Hispanic
	\item Middle Eastern
	\item African
	\item Caribbean
	\item South Asian
	\item East Asian
\end{itemize}
\textbf{In which country did you spend most of your life time?}\newline
(Dropdown menu)\medskip \newline

\noindent \textbf{How do you self-assess your level of English language?}\newline
We use the Common European Framework of Reference for Languages (CEFR) scale:
\begin{itemize}
	\item C2	Proficient
	\item C1	Advanced
	\item B2	Upper-intermediate
	\item B1	Intermediate
	\item A2	Pre-intermediate
	\item A1	Elementary
	\item A0-A1	Beginner
\end{itemize}

\section{Appendix}


\subsection{Results Demographic Items}

\begin{table}[H]
	\centering
	\begin{tabular}{l|rr}
		\textbf{Gender} & \textbf{Absolute} & \textbf{Relative} \\ \midrule
		Male & 125 & 43.55 \%\\
		Female & 160 & 55.75 \%\\
		Inter / diverse & 2 & 0.70 \%\\ \bottomrule
	\end{tabular}
	\caption{Gender distribution of valid cases}
\end{table}

\begin{table}[H]
	\centering
	\begin{tabular}{l|rr}
		\textbf{Age} & \textbf{Absolute} & \textbf{Relative} \\ \midrule
		18-30 & 192 & 66.90 \%\\
		31-40 & 64 & 22.30 \%\\
		41-50 & 22 & 7.67 \%\\ 
		50+   & 9 & 3.14 \%\\ \bottomrule
	\end{tabular}
	\caption{Age distribution of valid cases}
\end{table}


\begin{table}[H]
	\centering
	\begin{tabular}{l|rr}
		\textbf{Ethnicity} & \textbf{Absolute} & \textbf{Relative} \\ \midrule
		Caucasian 			& 253 & 88.15 \%\\
		Latino / Hispanic 	& 12 & 4.18 \%\\
		Middle Eastern 		& 3 & 1.05 \%\\ 
		African 			& 5 & 1.74 \%\\ 
		Caribbean 			& 3 & 1.05 \%\\ 
		South Asian 		& 8 & 2.79 \%\\ 
		East Asian 			& 3 & 1.05 \%\\ \bottomrule
	\end{tabular}
	\caption{Ethnicity distribution of valid cases}
\end{table}


\begin{table}[H]
	\centering
	\begin{tabular}{l|rr}
		\textbf{Language Level} & \textbf{Absolute} & \textbf{Relative} \\ \midrule
		C2 			& 163 & 56.79 \%\\
		C1		 	& 68 & 23.69 \%\\
		B2	 		& 40 & 13.94 \%\\ 
		B1 			& 12 & 4.18 \%\\ 
		A2 			& 3 & 1.05 \%\\ 
		A1	 		& 1 & 0.35 \%\\ 
		A0 			& 0 & 0.00 \%\\ \bottomrule
	\end{tabular}
	\caption{English language level distribution of valid cases}
\end{table}




\subsection{Participant Distribution over Conditions}
\begin{table}[H]
	\centering
	\begin{tabular}{l|rrrrrrrrrr|r}
		\textbf{Condition} & \multicolumn{10}{l|}{\textbf{Set}} & \textbf{Total} \\
		  & 0 & 1 & 2 & 3 & 4 & 5 & 6 & 7 & 8 & 9 & \\ \midrule
		super-good & 	2 &  3 &  5 &  3 &  2 &  3 &  4 &  1 &  3 &  4 &  	 30 \\
		super-rand & 	1 &  2 &  1 &  1 &  2 &  2 &  2 &  8 &  2 &  11 &  	 32 \\
		super-no & 	6 &  1 &  3 &  1 &  4 &  3 &  2 &  3 &  6 &  5 &  	 34 \\
		medium-good & 	3 &  1 &  3 &  2 &  5 &  3 &  1 &  7 &  2 &  6 &  	 33 \\
		medium-rand & 	0 &  4 &  3 &  5 &  2 &  2 &  3 &  2 &  8 &  1 &  	 30 \\
		medium-no & 	1 &  1 &  1 &  1 &  4 &  1 &  9 &  4 &  5 &  3 &  	 30 \\
		bad-good & 	3 &  3 &  4 &  6 &  1 &  4 &  3 &  3 &  7 &  4 &  	 38 \\
		bad-rand & 	5 &  4 &  1 &  5 &  4 &  3 &  1 &  2 &  4 &  1 &  	 30 \\
		bad-no & 	2 &  1 &  3 &  0 &  2 &  9 &  4 &  6 &  2 &  1 &  	 30 \\ \bottomrule
	\end{tabular}
	\caption{Overview of valid cases over conditions and subsets}
\end{table}


\subsection{Visually Impaired Friendly Colour Scale}

\begin{figure}[H]
	\makebox[\textwidth][c]{
		\begin{minipage}[t]{0.65\textwidth}
			\centering
			\includegraphics[width=\textwidth]{img/results_matrix_understanding3.JPG}
			\caption{Comparison of perceived understanding scores ordered by classifier, value reporting difference of means ($\bar{x}_{row} - \bar{x}_{column}$ ), asterisk reporting significance (* significant at $\alpha=0.05$, ** significant at $\alpha=0.01$)}
		\end{minipage}%
		\hspace{5mm}
		\begin{minipage}[t]{0.65\textwidth}
			\centering
			\includegraphics[width=\textwidth]{img/results_matrix_understanding_reordered3.JPG}
			\caption{Comparison of perceived understanding scores ordered by explanation type, value reporting difference of means ($\bar{x}_{row} - \bar{x}_{column}$ ), asterisk reporting significance (* significant at $\alpha=0.05$, ** significant at $\alpha=0.01$)}
	\end{minipage}}
\end{figure}
\vspace{-5mm}
\begin{figure}[H]
	\makebox[\textwidth][c]{
		\begin{minipage}[t]{0.65\textwidth}
			\centering
			\includegraphics[width=\textwidth]{img/results_matrix_trust3.JPG}
			\caption{Comparison of trust scores ordered by classifier, value reporting difference of means ($\bar{x}_{row} - \bar{x}_{column}$ ), asterisk reporting significance (* significant at $\alpha=0.05$, ** significant at $\alpha=0.01$)}
		\end{minipage}%
		\hspace{5mm}
		\begin{minipage}[t]{0.65\textwidth}
			\centering
			\includegraphics[width=\textwidth]{img/results_matrix_trust_reordered3.JPG}
			\caption{Comparison of trust scores ordered by explanation type, value reporting difference of means ($\bar{x}_{row} - \bar{x}_{column}$ ), asterisk reporting significance (* significant at $\alpha=0.05$, ** significant at $\alpha=0.01$)}
	\end{minipage}}
\end{figure}
\vspace{-5mm}
\begin{figure}[H]
	\makebox[\textwidth][c]{
		\begin{minipage}[t]{0.65\textwidth}
			\centering
			\includegraphics[width=\textwidth]{img/results_matrix_proxy_away3.JPG}
			\caption{Comparison of proxy trust (away) scores ordered by classifier, value reporting difference of means ($\bar{x}_{row} - \bar{x}_{column}$ ), asterisk reporting significance (* significant at $\alpha=0.05$, ** significant at $\alpha=0.01$)}
		\end{minipage}%
		\hspace{5mm}
		\begin{minipage}[t]{0.65\textwidth}
			\centering
			\includegraphics[width=\textwidth]{img/results_matrix_proxy_away_reordered3.JPG}
			\caption{Comparison of proxy trust (away) scores ordered by explanation type, value reporting difference of means ($\bar{x}_{row} - \bar{x}_{column}$ ), asterisk reporting significance (* significant at $\alpha=0.05$, ** significant at $\alpha=0.01$)}
	\end{minipage}}
\end{figure}
\vspace{-5mm}
\begin{figure}[H]
	\makebox[\textwidth][c]{
		\begin{minipage}[t]{0.65\textwidth}
			\centering
			\includegraphics[width=\textwidth]{img/results_matrix_proxy_towards3.JPG}
			\caption{Comparison of proxy trust (towards) scores ordered by classifier, value reporting difference of means ($\bar{x}_{row} - \bar{x}_{column}$ ), asterisk reporting significance (* significant at $\alpha=0.05$, ** significant at $\alpha=0.01$)}
		\end{minipage}%
		\hspace{5mm}
		\begin{minipage}[t]{0.65\textwidth}
			\centering
			\includegraphics[width=\textwidth]{img/results_matrix_proxy_towards_reordered3.JPG}
			\caption{Comparison of proxy trust (towards) scores ordered by explanation type, value reporting difference of means ($\bar{x}_{row} - \bar{x}_{column}$ ), asterisk reporting significance (* significant at $\alpha=0.05$, ** significant at $\alpha=0.01$)}
	\end{minipage}}
\end{figure}
\vspace{-5mm}
\begin{figure}[H]
	\makebox[\textwidth][c]{
		\begin{minipage}[t]{0.65\textwidth}
			\centering
			\includegraphics[width=\textwidth]{img/results_matrix_predictability3.JPG}
			\caption{Comparison of predictability scores ordered by classifier, value reporting difference of means ($\bar{x}_{row} - \bar{x}_{column}$ ), asterisk reporting significance (* significant at $\alpha=0.05$, ** significant at $\alpha=0.01$)}
		\end{minipage}%
		\hspace{5mm}
		\begin{minipage}[t]{0.65\textwidth}
			\centering
			\includegraphics[width=\textwidth]{img/results_matrix_predictability_reordered3.JPG}
			\caption{Comparison of predictability scores ordered by explanation type, value reporting difference of means ($\bar{x}_{row} - \bar{x}_{column}$ ), asterisk reporting significance (* significant at $\alpha=0.05$, ** significant at $\alpha=0.01$)}
	\end{minipage}}
\end{figure}




\newpage
\subsection{Confusion Matrices Proxy Measure (Absolute Counts)}
\begin{figure}[H]
	\makebox[\textwidth][c]{
		\begin{minipage}[t]{0.65\textwidth}
			\centering
			\begin{table}[H]
				\centering
				\begin{tabular}{ll|rr}
					\multicolumn{2}{l|}{} & \multicolumn{2}{c}{\textbf{Truth}}\\
					\multicolumn{2}{l|}{} & Towards & Away \\ \midrule
					\multirow{2}{*}{\textbf{Classifier}} & Towards & 32 & 4 \\
					& Away & 0 & 6 \\ \bottomrule
				\end{tabular}
				\caption{Super-good}
			\end{table}
		\end{minipage}%
		\hspace{5mm}
		\begin{minipage}[t]{0.65\textwidth}
			\centering
			\begin{table}[H]
				\centering
				\begin{tabular}{ll|rr}
					\multicolumn{2}{l|}{} & \multicolumn{2}{c}{\textbf{Truth}}\\
					\multicolumn{2}{l|}{} & Towards & Away \\ \midrule
					\multirow{2}{*}{\textbf{Classifier}} & Towards & 22 & 2 \\
					& Away & 0 & 11 \\ \bottomrule
				\end{tabular}
				\caption{Super-rand}
			\end{table}
	\end{minipage}}
\end{figure}
\vspace{-2mm}
\begin{figure}[H]
	\makebox[\textwidth][c]{
		\begin{minipage}[t]{0.65\textwidth}
			\centering
			\begin{table}[H]
				\centering
				\begin{tabular}{ll|rr}
					\multicolumn{2}{l|}{} & \multicolumn{2}{c}{\textbf{Truth}}\\
					\multicolumn{2}{l|}{} & Towards & Away \\ \midrule
					\multirow{2}{*}{\textbf{Classifier}} & Towards & 31 & 1 \\
					& Away & 0 & 8 \\ \bottomrule
				\end{tabular}
				\caption{Super-no}
			\end{table}
		\end{minipage}%
		\hspace{5mm}
		\begin{minipage}[t]{0.65\textwidth}
			\centering
			\begin{table}[H]
				\centering
				\begin{tabular}{ll|rr}
					\multicolumn{2}{l|}{} & \multicolumn{2}{c}{\textbf{Truth}}\\
					\multicolumn{2}{l|}{} & Towards & Away \\ \midrule
					\multirow{2}{*}{\textbf{Classifier}} & Towards & 22 & 10 \\
					& Away & 3 & 2 \\ \bottomrule
				\end{tabular}
				\caption{Medium-good}
			\end{table}
	\end{minipage}}
\end{figure}
\vspace{-2mm}
\begin{figure}[H]
	\makebox[\textwidth][c]{
		\begin{minipage}[t]{0.65\textwidth}
			\centering
			\begin{table}[H]
				\centering
				\begin{tabular}{ll|rr}
					\multicolumn{2}{l|}{} & \multicolumn{2}{c}{\textbf{Truth}}\\
					\multicolumn{2}{l|}{} & Towards & Away \\ \midrule
					\multirow{2}{*}{\textbf{Classifier}} & Towards & 17 & 9 \\
					& Away & 3 & 12 \\ \bottomrule
				\end{tabular}
				\caption{Medium-rand}
			\end{table}
		\end{minipage}%
		\hspace{5mm}
		\begin{minipage}[t]{0.65\textwidth}
			\centering
			\begin{table}[H]
				\centering
				\begin{tabular}{ll|rr}
					\multicolumn{2}{l|}{} & \multicolumn{2}{c}{\textbf{Truth}}\\
					\multicolumn{2}{l|}{} & Towards & Away \\ \midrule
					\multirow{2}{*}{\textbf{Classifier}} & Towards & 20 & 7 \\
					& Away & 1 & 2 \\ \bottomrule
				\end{tabular}
				\caption{Medium-no}
			\end{table}
	\end{minipage}}
\end{figure}
\vspace{-2mm}
\begin{figure}[H]
	\makebox[\textwidth][c]{
		\begin{minipage}[t]{0.65\textwidth}
			\centering
			\begin{table}[H]
				\centering
				\begin{tabular}{ll|rr}
					\multicolumn{2}{l|}{} & \multicolumn{2}{c}{\textbf{Truth}}\\
					\multicolumn{2}{l|}{} & Towards & Away \\ \midrule
					\multirow{2}{*}{\textbf{Classifier}} & Towards & 1 & 27 \\
					& Away & 7 & 1 \\ \bottomrule
				\end{tabular}
				\caption{Bad-good}
			\end{table}
		\end{minipage}%
		\hspace{5mm}
		\begin{minipage}[t]{0.65\textwidth}
			\centering
			\begin{table}[H]
				\centering
				\begin{tabular}{ll|rr}
					\multicolumn{2}{l|}{} & \multicolumn{2}{c}{\textbf{Truth}}\\
					\multicolumn{2}{l|}{} & Towards & Away \\ \midrule
					\multirow{2}{*}{\textbf{Classifier}} & Towards & 1 & 23 \\
					& Away & 10 & 0 \\ \bottomrule
				\end{tabular}
				\caption{Bad-rand}
			\end{table}
	\end{minipage}}
\end{figure}
\vspace{-2mm}
\begin{table}[H]
	\centering
	\begin{tabular}{ll|rr}
		\multicolumn{2}{l|}{} & \multicolumn{2}{c}{\textbf{Truth}}\\
		\multicolumn{2}{l|}{} & Towards & Away \\ \midrule
		\multirow{2}{*}{\textbf{Classifier}} & Towards & 2 & 31 \\
		& Away & 4 & 0 \\ \bottomrule
	\end{tabular}
	\caption{Bad-no}
\end{table}


%\newpage
%\subsection{Confusion Matrices Proxy Measure (Normalised over opportunities)}
%\vspace{-5mm}


\end{document}
